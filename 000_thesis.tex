%%%%%%%%%%%%%%%%%%%%%%%%%%%%%%%%%%%%%%%%%%%%%%%%%%%%%%%%%%%%%%%%%%%%%%%%%
%                                                                       %
% ustthesis_test.tex: A template file for usage with ustthesis.cls      %
%                                                                       %
%%%%%%%%%%%%%%%%%%%%%%%%%%%%%%%%%%%%%%%%%%%%%%%%%%%%%%%%%%%%%%%%%%%%%%%%%

\documentclass[a4paper]{ustthesis}
\usepackage[square,numbers]{natbib}
\usepackage{xeCJK}
\usepackage[T1]{fontenc}
%\usepackage{algorithm}
%\usepackage{algorithmic}
\usepackage{longtable}
\usepackage{url}
\usepackage{multirow}
\usepackage{graphics}
\usepackage{graphicx}
\usepackage{amsmath}
\usepackage{cases}
\usepackage{colortbl}
\usepackage{xcolor}
\usepackage{times}
\usepackage{array}
%\usepackage{hyperref}
\usepackage{pdflscape}
\DeclareMathOperator*{\argmax}{argmax}      % for argmax

% \usepackage{latexsym}
    % Use the "latexsym" package when encountering the following error:
    %   ! LaTeX Error: Command \??? not provided in base LaTeX2e.
% \usepackage{epsf}
    % Use the "epsf" package for including EPS files.

%%%%%%%%%%%%%%%%%%%%%%%%%%%%%%%%%%%%%%%%%%%%%%%%%%%%%%%%%%%%%%%%%%%%%%%%%
%                                                                       %
% Preambles. DO NOT ERASE THEM. Change to suite your particular purpose.%
%                                                                       %
%%%%%%%%%%%%%%%%%%%%%%%%%%%%%%%%%%%%%%%%%%%%%%%%%%%%%%%%%%%%%%%%%%%%%%%%%

\title{Quasi-Two-Dimensional Coulomb Systems: Modeling, Simulation, and Applications}  % Title of the thesis.
\author{GAO, Xuanzhao}     % Author of the thesis.
\degree{\PhD}             % Degree for which the thesis is. Options: \AM \MSc \MPhil \PhD
\stage{\Thesis}              % Stage of PhD document; use \Thesis for all other degree. Options: \PQE \Proposal \Thesis
\subject{PhD Individual Interdisci Prog} % Subject of the Degree.
\department{The Advanced Material Thrust}       % Department to which the thesis is submitted.
\advisor{GAN, Zecheng}     % Supervisor. Additional co-supervisor can be added using \member
\member{XIANG, Yang, Thesis Co-Supervisor}
\member{LIU, Jinguo, Thesis Co-Supervisor}
%\acting      % Uncomment for Accting department head
\depthead{GAO, Ping}     % department head.
\defencedate{2025}{06}{01}     % \defencedate{year}{month}{day}.

% NOTE:
%   According to the sample shown in the guidelines, page number is
%   placed below the bottom margin.  However, if the author prefers
%   the page number to be printed above the bottom margin, please
%   activate the following command.

%\PNumberAboveBottomMargin

\begin{document}
%\begin{CJK}{UTF8}{song}  % Bitstream Cyber Bit song ti

%\begin{CJK*}{UTF8}{gbsn} % Arphic song ti

%%%%%%%%%%%%%%%%%%%%%%%%%%%%%%%%%%%%%%%%%%%%%%%%%%%%%%%%%%%%%%%%%%%%%%%%%
%                                                                       %
% Now the actual Thesis. The order of output MUST be followed:          %
%                                                                       %
%    1) TITLEPAGE                                                       %
%                                                                       %
% The \maketitle command generates the Title page as well as the        %
% Signature page.                                                       %
%                                                                       %
%%%%%%%%%%%%%%%%%%%%%%%%%%%%%%%%%%%%%%%%%%%%%%%%%%%%%%%%%%%%%%%%%%%%%%%%%

\maketitle

%%%%%%%%%%%%%%%%%%%%%%%%%%%%%%%%%%%%%%%%%%%%%%%%%%%%%%%%%%%%%%%%%%%%%%%%%
%                                                                       %
%     2) DEDICATION (Optional)                                          %
%                                                                       %
% The \dedication and \enddedication commands are optional. If          %
% specified it generates a page for dedication.                         %
%
%%%%%%%%%%%%%%%%%%%%%%%%%%%%%%%%%%%%%%%%%%%%%%%%%%%%%%%%%%%%%%%%%%%%%%%%%

% \dedication
% % This is an optional section.
% \noindent You raise me up, so I can stand on mountains;\\
you raise me up, to walk on stormy seas.\\
I am strong when I am on your shoulders.\\
You raise me up, to more than I can be. --- ``You raise me up'' lyrics by Brendan Graham.\\

\par \hfill To my mother, CHAN Siu-ngan (1958.12.21 - 2017.12.12).\\

\par ~\\

\par
\noindent 找一個夢 走一條路 \\
你像月光夜夜在祈禱我幸福 照亮我踏的每一步\\
我不會哭 我不會輸 \\
我在月光守護的黑夜裡\\
看著自己真的像你 走你走過的路 --- 《月光》 易家揚 詞\\

\par \hfill 致陳笑顏(1958.12.21 - 2017.12.12),我敬愛的媽媽

% \enddedication
% \newpage

%%%%%%%%%%%%%%%%%%%%%%%%%%%%%%%%%%%%%%%%%%%%%%%%%%%%%%%%%%%%%%%%%%%%%%%%%
%                                                                       %
%     3) ACKNOWLEDGMENTS                                                %
%                                                                       %
% \acknowledgments and \endacknowledgments defines the                  %
% Acknowledgments of the author of the Thesis.                          %
%                                                                       %
%%%%%%%%%%%%%%%%%%%%%%%%%%%%%%%%%%%%%%%%%%%%%%%%%%%%%%%%%%%%%%%%%%%%%%%%%

\acknowledgments
% acknowledgments

Completing this thesis is a long journey, and I would like to thank everyone who has helped me along the way.

First and foremost, I would like to express my deepest gratitude to my advisor, Prof. Zecheng Gan from Hong Kong University of Science and Technology (Guangzhou), under whose guidance the main work of this thesis was completed.
In the summer of 2021, I became Prof. Gan's first PhD student, marking the beginning of our journey together.
During my undergraduate studies, I majored in Applied Physics. 
Although I chose to enter this entirely new field due to my interest in applied mathematics and scientific computing, I had not received sufficient training in these areas.
However, Prof. Gan never dismissed me for my lack of knowledge. Instead, he patiently tutored me in the fundamentals of scientific computing and applied mathematics, and provided guidance and support whenever I encountered difficulties.
Under these circumstances, I was able to quickly begin our first project and make progress in my academic growth.
During my PhD study, I often felt inadequate: I wasn't particularly clever, having been just an average student during my undergraduate years; I wasn't meticulous enough, often making mistakes due to carelessness; and I wasn't always diligent, frequently being distracted.
Yet Prof. Gan never criticized me for my shortcomings. 
Instead, he consistently encouraged me, supported me, and helped me improve.
Prof. Gan also gave me considerable academic freedom, allowing me to explore different ideas and directions. 
It was through this exploration that I gradually found the direction I was willing to dedicate myself to, which has shaped who I am today.

Second, I would like to express my gratitude to several other professors who have provided invaluable guidance and support throughout my research journey. 
I am particularly thankful to Prof. Zhenli Xu from Shanghai Jiao Tong University. 
During my first semester of junior year, I had the privilege of being a visiting student in Prof. Xu's group.
Although one semester may seem brief, Prof. Xu's rigorous research approach and passion for science left a lasting impression on me. 
This period marked a significant transformation, as I evolved from a naive student into a researcher with a deep enthusiasm for scientific inquiry. Prof. Xu's comprehensive understanding of the field has been instrumental in guiding my future research direction.
I also deeply appreciate Prof. Jin-Guo Liu of Hong Kong University of Science and Technology (Guangzhou), my co-supervisor. 
Although our relationship did not begin as advisor and student, he generously shared his knowledge with me. His innovative research mindset and professional programming skills have been immensely beneficial. 
Collaborating with Prof. Liu on open-source projects and research work has been both educational and enjoyable. The coding work for this thesis would have been challenging to complete without his assistance.
I am grateful to Prof. Shidong Jiang from the Flatiron Institute, with whom I have had numerous discussions over the past year. 
Prof. Jiang's ability to address the core issues has significantly enhanced the quality of my work.
Additionally, I would like to thank Prof. Yang Xiang from Hong Kong University of Science and Technology, Prof. Pan Zhang from the Institute of Theoretical Physics, Chinese Academy of Sciences, and Prof. Feng Pan from Singapore University of Technology and Design. 
Their support has been crucial in overcoming various challenges during my thesis, enabling me to complete my doctoral research on schedule.

I would also like to thank my collaborators, including Jiuyang Liang from the Flatiron Institute, Qi Zhou from Shanghai Jiao Tong University, and Yijia Wang from the Institute of Theoretical Physics, Chinese Academy of Sciences. 
The time spent with them has always been enjoyable, and discussions with them have always brought me new knowledge and ideas. 
Most of my work has been completed in collaboration with them. 
Whether as collaborators or friends, they are impeccable. 
Now we are all ready to embark on our own paths, and I hope that in the future, we will all shine in our respective fields.
I also want to thank my friends, Zheng Yang, Tianhao Hu, Yanyu Duan, Yusheng Zhao, Zhongyi Ni, and Hongchao Li. 
Without their companionship, the life of a PhD student would have been more difficult.

Finally, I would like to thank my parents for their unwavering support and encouragement.
Over the past four years, our family has faced many challenges, but they have shouldered all the burdens alone, just to keep me from being distracted.
Their unconditional support and love are the driving forces behind my progress.

This is indeed a long journey, but it is also only the beginning of my life as a scientist.
Knowledge begins with the recognition of one's ignorance. 
The realization that the search for knowledge is unending.
More challenges await me, and I will continue to explore the world with curiosity and passion, carrying forward the spirit of scientific inquiry.


\section*{A note of publications}

Most of the material in this thesis has already appeared in the following peer-reviewed publications or articles under review:
\begin{itemize}
    \item[A.] \textbf{Xuanzhao Gao}, Qi Zhou, Zecheng Gan, Jiuyang Liang; Accurate error estimates and optimal parameter selection in Ewald summation for dielectrically confined Coulomb systems. ArXiv: 2503.18126.
    \item[B.] Zecheng Gan, \textbf{Xuanzhao Gao}, Jiuyang Liang, Zhenli Xu; Random batch Ewald method for dielectrically confined Coulomb systems. Accepted by \emph{SIAM Journal on Scientific Computing}. ArXiv: 2405.06333.
    \item[C.] Zecheng Gan, \textbf{Xuanzhao Gao}, Jiuyang Liang, Zhenli Xu; Fast algorithm for quasi-2D Coulomb systems. \emph{Journal of Computational Physics}, 524: 113733, 2025.
    \item[D.] \textbf{Xuanzhao Gao}, Zecheng Gan; Broken symmetries in quasi-2D charged systems via negative dielectric confinement. \emph{The Journal of Chemical Physics}, 161 (1): 011102, 2024.
\end{itemize}
We also present some new material in this thesis, which will appear in a forthcoming publication:
\begin{itemize}
    \item[E.] \textbf{Xuanzhao Gao}, Zecheng Gan, Yuqing Li; Efficient particle-based simulations of Coulomb systems under dielectric nanoconfinement.
\end{itemize}
In particular, the material of Chapter~\ref{chp_icmewald2d} comes from A, where we present our theoretical analysis of the Ewald splitting method for confined quasi-2D Coulomb systems.
Then in Chapters~\ref{chp_soewald2d},~\ref{chp_rbe2d}, and~\ref{chp_quasiewald}, we present our fast algorithms for quasi-2D Coulomb systems, which are contributions from B, C, and E, respectively.
In Chapter~\ref{chp_applications}, we present results on numerical experiments on confined quasi-2D Coulomb systems from B and D.
\endacknowledgments
\newpage

%%%%%%%%%%%%%%%%%%%%%%%%%%%%%%%%%%%%%%%%%%%%%%%%%%%%%%%%%%%%%%%%%%%%%%%%%
%                                                                       %
%     4) TABLE OF CONTENTS                                              %
%                                                                       %
%%%%%%%%%%%%%%%%%%%%%%%%%%%%%%%%%%%%%%%%%%%%%%%%%%%%%%%%%%%%%%%%%%%%%%%%%

\tableofcontents

%%%%%%%%%%%%%%%%%%%%%%%%%%%%%%%%%%%%%%%%%%%%%%%%%%%%%%%%%%%%%%%%%%%%%%%%%
%                                                                       %
%     5) LIST OF FIGURES (If Any)                                       %
%                                                                       %
%%%%%%%%%%%%%%%%%%%%%%%%%%%%%%%%%%%%%%%%%%%%%%%%%%%%%%%%%%%%%%%%%%%%%%%%%

\listoffigures

%%%%%%%%%%%%%%%%%%%%%%%%%%%%%%%%%%%%%%%%%%%%%%%%%%%%%%%%%%%%%%%%%%%%%%%%%
%                                                                       %
%     6) LIST OF TABLES (If Any)
%                                                                       %
%%%%%%%%%%%%%%%%%%%%%%%%%%%%%%%%%%%%%%%%%%%%%%%%%%%%%%%%%%%%%%%%%%%%%%%%%

\listoftables

%%%%%%%%%%%%%%%%%%%%%%%%%%%%%%%%%%%%%%%%%%%%%%%%%%%%%%%%%%%%%%%%%%%%%%%%%
%                                                                       %
%     7) ABSTRACT                                                       %
%                                                                       %
% \abstract and \endabstract are used to define a short Abstract for    %
% the Thesis.                                                           %
%                                                                       %
%%%%%%%%%%%%%%%%%%%%%%%%%%%%%%%%%%%%%%%%%%%%%%%%%%%%%%%%%%%%%%%%%%%%%%%%%

\abstract
Abstract text, 300 words or less
\endabstract

%%%%%%%%%%%%%%%%%%%%%%%%%%%%%%%%%%%%%%%%%%%%%%%%%%%%%%%%%%%%%%%%%%%%%%%%%
%                                                                       %
%     8) The Actual Contents                                            %
%                                                                       %
% The command \chapters MUST BE USED to ensure that the entire content  %
% of the Thesis is double-spaced (in version 1.0).                      %
%                                                                       %
% However, in version 2.0, \chapters will be automatically added in     %
% the beginning of the first chapter.                                   %
%                                                                       %
%%%%%%%%%%%%%%%%%%%%%%%%%%%%%%%%%%%%%%%%%%%%%%%%%%%%%%%%%%%%%%%%%%%%%%%%%

%%\chapters         % Not necessary with ustthesis.cls (v2.0).

%%%%%%%%%%%%%%%%%%%%%%%%%%%%%%%%%%%%%%%%%%%%%%%%%%%%%%%%%%%%%%%%%%%%%%%%%
%                                                                       %
% Each chapter is defined via the \chapter command. The usual sectional %
% commands of LaTeX are also available.                                 %
%                                                                       %
%%%%%%%%%%%%%%%%%%%%%%%%%%%%%%%%%%%%%%%%%%%%%%%%%%%%%%%%%%%%%%%%%%%%%%%%%

\chapter{Introduction}
\label{chp_intro}
\section{Quasi-2D Coulomb Systems and Dielectric Confinement}

Quasi-2D Coulomb systems have emerged as systems of fundamental importance, attracting significant attention across numerous scientific domains in recent years~\cite{guidelines}.

\section{Ewald Summation}

\section{The Random Batch Sampling Algorithm}
\newpage

\chapter{Guidelines on Thesis Preparation}
\label{chp_background}
Here is a version of the guidelines on thesis preparation with minor edits. Please refer to the original document \citep{guidelines} for the most update and accurate information about thesis preparation, especially about the thesis submission protocol. 
\section{Introduction}
The guidelines in this document seek to ensure that theses are presented in a form suitable for library cataloging, preserving and access by users. The thesis will take its place in the library as a product of original thinking, research, and writing; its form must be comparable to other published works.

These guidelines cover the general rules of format and appearance. For content requirements, students should consult their Thesis Supervision Committee (TSC).

It is the student's responsibility to follow the requirements presented here. Thesis copies that do not meet these requirements will not be accepted.

Because of changes in requirements over time, students should not use existing library or departmental copies of theses as examples of current proper format.

\section{Originality}
\subsection{MPhil thesis}
An MPhil thesis shall consist of the student's own account of his/her investigations; be either a record of original work or an ordered, critical and thorough exposition of existing knowledge; be an integrated whole, presenting a coherent argument; give a critical assessment of the relevant literature, describe the method of research and its findings, and discuss those findings; and include a full bibliography.
\subsection{PhD thesis}
A PhD thesis shall consist of the student's own account of his/her investigations; make original, distinct contribution(s) to our knowledge of the subject and afford evidence or originality by the discovery of new facts and/or by the exercise of independent critical power; be an integrated whole with a coherent argument; give a critical assessment of the relevant literature, describe the method of research and its findings, and discuss those findings, particularly with regard to how these findings appear to the candidate to have advanced the study of the subject; include a full bibliography; and be of a standard to merit publication in whole, in part or in a revised form (for example, as a monograph or as a number of articles in learned journals).

\section{Components}
\subsection{Order}
A thesis should contain a Title page (containing thesis title, full name of the candidate, degree for which the thesis is submitted, name of the University, i.e. The Hong Kong University of Science and Technology, month and year of submission), Authorization page, Signature page, Acknowledgments, Table of contents, Lists of figures and tables, Abstract, Thesis body, Bibliography, Appendices and other addenda, if any.
\subsection{Authorization page}
On this page, students authorize the University to lend or reproduce the thesis. The copyright of the thesis as a literary work vests in its author (the student). The authorization gives HKUST Library a non-exclusive right to make it available for scholarly research.
\subsection{Signature page}
This page provides signatures of the thesis supervisor(s) and Department Head confirming that the thesis is satisfactory.
\subsection{Acknowledgments}
The student is required to declare, in this section, the extent to which assistance has been given by his/her faculty and staff, fellow students, external bodies or others in the collection of materials and data, the design and construction of apparatus, the performance of experiments, the analysis of data, and the preparation of the thesis (including editorial help). In addition, it is appropriate to recognize the supervision and advice given by the thesis supervisor(s) and members of TSC.
\subsection{Abstract}
Every copy of the thesis must have an English abstract, being a concise summary of the thesis, in 300 words or less.
\subsection{Bibliography}
The list of sources and references used should be presented in a standard format appropriate to the discipline; formatting should be consistent throughout.
\subsection{Sample pages}
Sample of both MPhil \citep{mphil} and PhD \citep{phd} theses are provided, with specific instructions for formatting page content (centering, spacing, etc.).

\section{Language, Style and Format}
\subsection{Language}
Theses should be written in English.

Students in the School of Humanities and Social Science who are pursuing research work in the areas of Chinese Studies, and who can demonstrate a need to use Chinese to write their theses should seek prior approval from the School via their thesis supervisor and the divisional head. If approval is granted, students are also required to produce a translation of the title page, authorization page, signature page, table of contents and the abstract in English.

\subsection{Pagination}
All pages, starting with the Title page should be numbered. All page numbers should be centered, at the bottom of each page.

Page numbers of materials preceding the body of the text should be in small Roman numerals. Page numbers of the text, beginning with the first page of the first chapter and continuing through the bibliography, including any pages with tables, maps, figures, photographs, etc., and any subsequent appendices, should be in Arabic numerals.\footnote{That means the Title page will be page i; the first page of the first chapter will be page 1.}

Start a new page after each chapter or section but not after a sub-section.

\subsection{Format}
A conventional font, size 12-point, 10 to 12 characters per inch must be used. One-and-a-half line spacing should be used throughout the thesis, except for abstracts, indented quotations or footnotes where single line spacing may be used.

All margins---top, bottom, sides---should be consistently 25mm (or no more than 30mm) in width. The same margin should be used throughout a thesis. Exceptionally, margins of a different size may be used when the nature of the thesis requires it.

\subsection{Footnotes}
Footnotes may be placed at the bottom of the page, at the end of each chapter or after the end of the thesis body. Like references, footnotes should be presented in a standard format appropriate to the discipline. Both the position and format of footnotes should be consistent throughout the thesis.

\subsection{Appendices}
The format of each appended item should be consistent with the nature of that item, whether text, diagram, figure, etc., and should follow the guidelines for that item as listed here.

\subsection{Figures, Tables and Illustrations}
Figures, tables, graphs, etc., should be positioned according to the scientific publication conventions of the discipline, e.g., interspersed in text or collected at the end of chapters. Charts, graphs, maps, and tables that are larger than a standard page should be provided as appendices.

\subsection{Photographs/Images}
High contrast photos should be used because they reproduce well. Photographs with a glossy finish and those with dark backgrounds should be avoided. Images should be dense enough to provide 300 ppi for printing and 72 dpi for viewing.

\subsection{Additional Materials}
Raw files, datasets, media files, and high resolution photographs/images of any format can be included. \footnote{Students should get approval from their Department Head before deviating from any of the above requirements concerning paper size, font, margins, etc.}

\section{Creating PDF files}
Theses must be submitted in PDF format. Providing a properly generated PDF file ensures the manuscript can be read using different platforms (Windows, Mac, etc.), that it displays as intended, and that it can be readily indexed.

Before submitting the PDF file, please use the the HKUST E-thesis PDF Preflight application to ensure all fonts should be embedded; image resolution should be dense enough to provide 300 ppi for printing and 72 dpi for viewing; the final thesis should be submitted as a single PDF file and PDF files should NOT be encrypted, as text cannot be extracted from encrypted PDFs for full text indexing or storage. Encrypted PDF files will NOT be accepted.

\section{Thesis Submission Protocol}
The final thesis must be free from typographical, grammatical and other errors when submitted to the Thesis Submission System. In particular, the thesis supervisor and the Department Head/program director should not sign off on the final thesis that is not, to the best of their knowledge, free of errors.

For examination purpose, sufficient hard or electronic thesis copies are to be submitted to the Department at least four weeks before the thesis examination. The number of copies required will depend on the number of examiners.

Students should submit the draft thesis to the iThenticate platform for originality check. The draft thesis together with the iThenticate report should be submitted to the Department no less than four weeks before the thesis examination.

On successful completion of the thesis examination, and after any required corrections, students must submit a copy of the final thesis (either hard/electronic) to their Department, which will arrange for the appropriate signatures of approval to be obtained.
For final theses which have been graded “Passed subject to minor corrections” or “Passed subject to major corrections”, students are required to submit the thesis for originality check via iThenticate. The iThenticate report should be handed in to the thesis supervisor(s), and the Thesis Examination Committee if applicable, for review and endorsement via their Department.
The Department will then return the signed Signature Page to the candidate.
The candidate will upload and submit the Signature Page and the Authorization Page as a PDF file and the final thesis as another PDF file to the University’s Thesis Submission System. The candidate does not need to replace the two unsigned pages in the thesis PDF with the scanned signature pages. The final thesis will be forwarded to the thesis supervisor(s) for approval via the Thesis Submission System.

\section{Copyright}
According to the University’s Intellectual Property Policy, students shall own the copyright in respect of their written coursework, theses, papers and publications themselves as a whole as literary works.

\section{Thesis Binding}
Students may wish to keep personal copies of their thesis. They may arrange for such copies on their own and at their own expense. Service from MTPC of the University is one option (details below). Students may explore other binderies for the binding service. In any case, the binding of the thesis must correspond with the University regulations.

\newpage

%%%%%%%%%%%%%%%%%%%%%%%%%%%%%%%%%%%%%%%%%%%%%%%%%%%%%%%%%%%%%%%%%%%%%%%%%
%                                                                       %
%      9) BIBLIOGRAPHY                                                  %
%                                                                       %
% This example uses bibtex to generate the required Bibliography. Refer %
% to the % the file ustthesis_test.bib for the entries of the           %
% Bibliography. Note that only the cited entries are printed.           %
%                                                                       %
% If BibTeX is not used to typeset the bibliography, replace the        %
% following line with the \begin{thebibliography} and \end{bibliography}%
% commands (the "thebibliography" environment) to process the           %
% Bibliography.                                                         %
%                                                                       %
%%%%%%%%%%%%%%%%%%%%%%%%%%%%%%%%%%%%%%%%%%%%%%%%%%%%%%%%%%%%%%%%%%%%%%%%%

%%%%%%%%%%%%%%%%%%%%%%%%%%%%%%%%%%%%%%%%%%%%%%%%%%%%%%%%%%%%%%%%%%%%%%%%%
%                                                                       %
% The recommended bibliography style is the IEEE bibliography style.    %
% "ustbib" defines the IEEE bibliography standard with the added        %
% ability of sorting the items by name of author.                       %
%                                                                       %
% If you are not using BibTeX to process your Bibliography, comment out %
% the following line.                                                   %
%                                                                       %
%%%%%%%%%%%%%%%%%%%%%%%%%%%%%%%%%%%%%%%%%%%%%%%%%%%%%%%%%%%%%%%%%%%%%%%%%

\addcontentsline{toc}{chapter}{Bibliography}
\bibliographystyle{IEEEtran}
\bibliography{references}
\newpage

%%%%%%%%%%%%%%%%%%%%%%%%%%%%%%%%%%%%%%%%%%%%%%%%%%%%%%%%%%%%%%%%%%%%%%%%%
%                                                                       %
%     10) APPENDIX (If Any)                                              %
%                                                                       %
% \appendix command marks the beginning of the APPENDIX part of the     %
% Thesis. The usual \chapter command is used for the different chapters %
% of the Appendix.                                                      %
%                                                                       %
%%%%%%%%%%%%%%%%%%%%%%%%%%%%%%%%%%%%%%%%%%%%%%%%%%%%%%%%%%%%%%%%%%%%%%%%%
\appendix
\chapter{Tips on HKUST thesis preparation}
\label{chp_tips}
\section{Sideway page}
Here is an example to have a page turned side way in the PDF. It is useful for having wide figures and tables in the thesis. 

\begin{landscape} 
\begin{table}
\caption{Pearson's
        $\rho $        correlation of participated metrics with the WMT 2016 official average
        direct assessment human judgments on newstest 10k hybrid super-sampled
        systems at system level. Correlations of metrics not significantly
        outperformed by any other for that language pair are highlighted in
        bold.}
\centering

\begin{tabular}{m{0.36\textwidth}cccccccc|cccccccc}
\hline & \multicolumn{8}{c|}{\textbf{into-English}} & \multicolumn{8}{c}{\textbf{out-of-English}} \\\textbf{Metric} & \textbf{cs} & \textbf{de} & \textbf{fi} & \textbf{lv} & \textbf{ru} & \textbf{tr} & \textbf{zh} & \textbf{avg.} & \textbf{cs} & \textbf{de} & \textbf{fi} & \textbf{lv} & \textbf{ru} & \textbf{tr} & \textbf{zh} & \textbf{avg.} \\ \hline
Blend  & .963 & \textbf{.969} & .956 & .976 & \textbf{.957} & .981 & .890 & .956 & -- & -- & -- & -- & .950 & -- & -- & -- \\ \hline
BEER & .966 & .952 & .954 & .974 & .930 & .969 & .897 & .949 & .963 & .829 & .975 & .923 & .942 & .968 & .906 & .929 \\ \hline
UHH\_TSKM & .990 & .929 & .918 & \textbf{.986} & .908 & .982  & .896  & .944  & --  & --  & --  & --  & --  & --  & --  & --  \\ \hline
CDER & .983  & .922  & .925  & .981  & .916  & .970  & \textbf{.898 } & .942  & .958  & .803  & .962  & .911  & .922  & .948  & \textbf{.975 } & .926  \\ \hline
TreeAggreg & .977  & .913  & \textbf{.975} & .983  & .912  & \textbf{.983 } & .854  & .942  & .942  & .765  & .963  & .915  & .919  & .971  & .933  & .915  \\ \hline
chrF++ & .935  & .957  & .924  & .970  & .938  & .957  & .876  & .937  & .966  & .835  & .977  & .944  & .942  & .975  & .968  & .944  \\ \hline
chrF & .933  & .960  & .935  & .965  & .946  & .941  & .855  & .934  & .968  & .845  & \textbf{.979 } & .945  & .947  & \textbf{.980 } & .969  & .947  \\ \hline
NIST & \textbf{.994} & .917  & .928  & .957  & .904  & .969  & .831  & .929  & .954  & .761  & .957  & .914  & .917  & .976  & .968  & .921  \\ \hline
TER & .983  & .899  & .950  & .967  & .905  & .951  & .837  & .927  & .951  & .790  & .959 & .888 & .930 & .958 & .965 & .920 \\ \hline
BLEU & .964  & .914  & .906  & .974  & .907  & .969  & .852  & .927  & .945  & .793  & .919  & .839  & .893  & .916  & .969  & .896  \\ \hline
MEANT-dvw.cos.wmax.n2.m.r8.$\alpha $1.0.$\beta $0.1 & .921  & .942  & .939  & .967  & .955  & .931  & .836  & .927  & --  & .844  & --  & --  & --  & --  & .944  & --  \\ \hline
CharacTER & .963  & .965  & .944  & .927  & .948  & .946  & .740  & .919  & \textbf{.973 } & \textbf{.893 } & .970  & .892  & .929  & .961  & .914  & .933 \\ \hline
WER & .981  & .889  & .946  & .965  & .900  & .922  & .828  & .919  & .949  & .797  & .959  & .884  & .931  & .947  & .951  & .917  \\ \hline
PER & .967  & .920  & .892  & .958  & .904  & .898  & .866  & .915  & .960  & .680  & .939  & .817  & .876  & .955  & .893  & .874  \\ \hline
MEANT-dvw.cos.wmax.n2.m.r8.$\alpha $1.0.$\beta $0.0 & .896  & .928  & .931  & .960  & .952  & .895  & .799  & .909  & .968  & .753  & .970  & \textbf{.947 } & \textbf{.955 } & .980  & .931  & .929  \\ \hline
AutoDA & .440  & .951  & .922  & .970  & .902  & .914  & .734  & .833  & .967  & .602  & .879  & .731  & .850  & .586  & .968  & .797  \\ \hline
\end{tabular}
\label{tbl_wmt17-sys}
\end{table}
\end{landscape} 


%%%%%%%%%%%%%%%%%%%%%%%%%%%%%%%%%%%%%%%%%%%%%%%%%%%%%%%%%%%%%%%%%%%%%%%%%
%                                                                       %
%     11) BIOGRAPHY (optional)                                          %
%                                                                       %
% \biography and \endbiography are used to define the optional          %
% Biography of the author of the Thesis.                                %
%                                                                       %
%%%%%%%%%%%%%%%%%%%%%%%%%%%%%%%%%%%%%%%%%%%%%%%%%%%%%%%%%%%%%%%%%%%%%%%%%

\end{document}
