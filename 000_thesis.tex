%%%%%%%%%%%%%%%%%%%%%%%%%%%%%%%%%%%%%%%%%%%%%%%%%%%%%%%%%%%%%%%%%%%%%%%%%
%                                                                       %
% ustthesis_test.tex: A template file for usage with ustthesis.cls      %
%                                                                       %
%%%%%%%%%%%%%%%%%%%%%%%%%%%%%%%%%%%%%%%%%%%%%%%%%%%%%%%%%%%%%%%%%%%%%%%%%

\documentclass[a4paper]{ustthesis}
\usepackage[square,numbers]{natbib}
\usepackage[T1]{fontenc}
%\usepackage{algorithm}
%\usepackage{algorithmic}
\usepackage{longtable}
\usepackage{url}
\usepackage{multirow}
\usepackage{graphics}
\usepackage{graphicx}
\usepackage{amsmath}
\usepackage{cases}
\usepackage{colortbl}
\usepackage{xcolor}
\usepackage{times}
\usepackage{array}
%\usepackage{hyperref}
\usepackage{pdflscape}
\DeclareMathOperator*{\argmax}{argmax}      % for argmax

\usepackage{lipsum}
\usepackage{amsfonts}
\usepackage{amsfonts,amssymb,amsmath,mathrsfs}
\usepackage{graphicx}
\usepackage{epstopdf}
%\usepackage{algorithmic}
\usepackage{bm}

\ifpdf
  \DeclareGraphicsExtensions{.eps,.pdf,.png,.jpg}
\else
  \DeclareGraphicsExtensions{.eps}
\fi

\def\b{\boldsymbol}
\def\bs{\bm{\sigma}}
\def\e{\epsilon}
\def\z{\zeta}
\def\Z{\mathbb{Z}}
\def\R{\mathbb{R}}
\def\E{\mathbb{E}}
\def\bbP{\mathbb{P}}
\def\bfP{\mathbf{P}}
\def\bfX{\mathbf{X}}
\def\cL{\mathcal{L}}
\def\cO{\mathcal{O}}
\def\cH{\mathcal{H}}
\def\Ld{\Lambda}
\def\calk{\mathcal{K}}
\def\Ai{\mathrm{\AA}}
\DeclareMathOperator{\supp}{supp}
\DeclareMathOperator{\dist}{dist}
\DeclareMathOperator{\var}{var}
\DeclareMathOperator{\tr}{tr}
\DeclareMathOperator{\loc}{loc}
\DeclareMathOperator{\kl}{KL}
\DeclareMathOperator{\erf}{erf}
\DeclareMathOperator{\erfc}{erfc}
\def\Image{\mathrm{Im}}

\usepackage{geometry}
\geometry{left=3.5cm,right=3.5cm,top=3cm,bottom=4cm}

\usepackage{colortbl}
\usepackage{xcolor}
\newcommand{\todo}[1] {\textit{todo}:{\color{red}#1}}
\newcommand{\rev}[1] {{\color{blue}#1}}
\newcommand{\revtwo}[1] {{\color{orange}#1}}
% a new command rm to remove the content
\usepackage{soul}
\newcommand{\rem}[1] {{\color{red}{\st{#1}}}}
\usepackage{bm}
\newcommand\Wtilde{\stackrel{\sim}{\smash{\mathcal{W}}\rule{0pt}{1.1ex}}}

\usepackage{algorithm}
\usepackage{algorithmicx}
\usepackage{algpseudocode}

%% The amssymb package provides various useful mathematical symbols
%\usepackage{amssymb}
%\usepackage{pifont}
%\usepackage[nointegrals]{wasysym}
%\usepackage{amsfonts,amssymb,amsbsy,latexsym,amsmath,tabulary,graphicx,times,caption,fancyhdr}
%\usepackage[utf8]{inputenc}
%\usepackage{url,multirow,morefloats,floatflt,cancel,tfrupee}
%\usepackage{mathrsfs}
%\usepackage{graphicx}
\urlstyle{rm}
%macros from Aleks
\newcommand{\V}[1]{\boldsymbol{#1}} %# vector
\newcommand{\M}[1]{\boldsymbol{#1}} %# matrix
\newcommand{\Set}[1]{\mathbb{#1}} %# set
\newcommand{\D}[1]{\Delta#1} %# \D{t} for time step size
\renewcommand{\d}[1]{\delta#1} %# \d{t} for small increment
\newcommand{\norm}[1]{\left\Vert #1\right\Vert } % norm
\newcommand{\abs}[1]{\left|#1\right|} %abs

\newcommand{\grad}{\M{\nabla}} %gradient
\newcommand{\av}[1]{\left\langle #1\right\rangle } %take average
\makeatletter
\newcommand{\Biggg}{\bBigg@{3.5}}

\newcommand{\sM}[1]{\M{\mathcal{#1}}} %matrix in mathcal font
\newcommand{\dprime}{\prime\prime} % double prime
\global\long\def\i{\iota}
\renewcommand{\i}{\iota} %i for imaginary unit
%\renewcommand{\i}{\mathsf i} %i for imaginary unit
\newcommand{\follows}{\quad\Rightarrow\quad} %=>
%\usepackage{amsthm}
\newcommand{\eqd}{\overset{d}{=}} %=^d
\newcommand{\te}{\text}
%\newtheorem{theorem}{Theorem}
%\newtheorem{lemma}{Lemma}
%\newtheorem{remark}{Remark}
%\newtheorem{proof}{Proof}
\newcommand{\tcr}[1]{\textcolor{red}{Note: #1}}
\newcommand{\tcb}[1]{\textcolor{blue}{#1}}
\newcommand{\xz}[1]{\textcolor{violet}{#1}}

% Add a serial/Oxford comma by default.
\newcommand{\creflastconjunction}{, and~}

% Used for creating new theorem and remark environments
% \newsiamremark{remark}{Remark}
% \newsiamremark{hypothesis}{Hypothesis}
% \crefname{hypothesis}{Hypothesis}{Hypotheses}
% \newsiamthm{claim}{Claim}

% \usepackage{latexsym}
    % Use the "latexsym" package when encountering the following error:
    %   ! LaTeX Error: Command \??? not provided in base LaTeX2e.
% \usepackage{epsf}
    % Use the "epsf" package for including EPS files.

%%%%%%%%%%%%%%%%%%%%%%%%%%%%%%%%%%%%%%%%%%%%%%%%%%%%%%%%%%%%%%%%%%%%%%%%%
%                                                                       %
% Preambles. DO NOT ERASE THEM. Change to suite your particular purpose.%
%                                                                       %
%%%%%%%%%%%%%%%%%%%%%%%%%%%%%%%%%%%%%%%%%%%%%%%%%%%%%%%%%%%%%%%%%%%%%%%%%

\title{Fast Algorithms for Simulating Quasi-2D Coulomb Systems}  % Title of the thesis.
\author{GAO, Xuanzhao}     % Author of the thesis.
\degree{\PhD}             % Degree for which the thesis is. Options: \AM \MSc \MPhil \PhD
\stage{\Thesis}              % Stage of PhD document; use \Thesis for all other degree. Options: \PQE \Proposal \Thesis
\subject{IIP-AMAT} % Subject of the Degree.
\department{The Advanced Material Thrust}       % Department to which the thesis is submitted.
\advisor{GAN, Zecheng}     % Supervisor. Additional co-supervisor can be added using \member
% \member{XIANG, Yang, Thesis Co-Supervisor}
% \member{LIU, Jinguo, Thesis Co-Supervisor}
%\acting      % Uncomment for Accting department head
\depthead{GAO, Ping}     % department head.
\defencedate{2025}{06}{01}     % \defencedate{year}{month}{day}.

% NOTE:
%   According to the sample shown in the guidelines, page number is
%   placed below the bottom margin.  However, if the author prefers
%   the page number to be printed above the bottom margin, please
%   activate the following command.

%\PNumberAboveBottomMargin

\begin{document}
%\begin{CJK}{UTF8}{song}  % Bitstream Cyber Bit song ti

%\begin{CJK*}{UTF8}{gbsn} % Arphic song ti

%%%%%%%%%%%%%%%%%%%%%%%%%%%%%%%%%%%%%%%%%%%%%%%%%%%%%%%%%%%%%%%%%%%%%%%%%
%                                                                       %
% Now the actual Thesis. The order of output MUST be followed:          %
%                                                                       %
%    1) TITLEPAGE                                                       %
%                                                                       %
% The \maketitle command generates the Title page as well as the        %
% Signature page.                                                       %
%                                                                       %
%%%%%%%%%%%%%%%%%%%%%%%%%%%%%%%%%%%%%%%%%%%%%%%%%%%%%%%%%%%%%%%%%%%%%%%%%

\maketitle

%%%%%%%%%%%%%%%%%%%%%%%%%%%%%%%%%%%%%%%%%%%%%%%%%%%%%%%%%%%%%%%%%%%%%%%%%
%                                                                       %
%     2) DEDICATION (Optional)                                          %
%                                                                       %
% The \dedication and \enddedication commands are optional. If          %
% specified it generates a page for dedication.                         %
%
%%%%%%%%%%%%%%%%%%%%%%%%%%%%%%%%%%%%%%%%%%%%%%%%%%%%%%%%%%%%%%%%%%%%%%%%%

% \dedication
% % This is an optional section.
% \noindent You raise me up, so I can stand on mountains;\\
you raise me up, to walk on stormy seas.\\
I am strong when I am on your shoulders.\\
You raise me up, to more than I can be. --- ``You raise me up'' lyrics by Brendan Graham.\\

\par \hfill To my mother, CHAN Siu-ngan (1958.12.21 - 2017.12.12).\\

\par ~\\

\par
\noindent 找一個夢 走一條路 \\
你像月光夜夜在祈禱我幸福 照亮我踏的每一步\\
我不會哭 我不會輸 \\
我在月光守護的黑夜裡\\
看著自己真的像你 走你走過的路 --- 《月光》 易家揚 詞\\

\par \hfill 致陳笑顏(1958.12.21 - 2017.12.12),我敬愛的媽媽

% \enddedication
% \newpage

%%%%%%%%%%%%%%%%%%%%%%%%%%%%%%%%%%%%%%%%%%%%%%%%%%%%%%%%%%%%%%%%%%%%%%%%%
%                                                                       %
%     3) ACKNOWLEDGMENTS                                                %
%                                                                       %
% \acknowledgments and \endacknowledgments defines the                  %
% Acknowledgments of the author of the Thesis.                          %
%                                                                       %
%%%%%%%%%%%%%%%%%%%%%%%%%%%%%%%%%%%%%%%%%%%%%%%%%%%%%%%%%%%%%%%%%%%%%%%%%

\acknowledgments
% acknowledgments

Completing this thesis is a long journey, and I would like to thank everyone who has helped me along the way.

First and foremost, I would like to express my deepest gratitude to my advisor, Prof. Zecheng Gan from Hong Kong University of Science and Technology (Guangzhou), under whose guidance the main work of this thesis was completed.
In the summer of 2021, I became Prof. Gan's first PhD student, marking the beginning of our journey together.
During my undergraduate studies, I majored in Applied Physics. 
Although I chose to enter this entirely new field due to my interest in applied mathematics and scientific computing, I had not received sufficient training in these areas.
However, Prof. Gan never dismissed me for my lack of knowledge. Instead, he patiently tutored me in the fundamentals of scientific computing and applied mathematics, and provided guidance and support whenever I encountered difficulties.
Under these circumstances, I was able to quickly begin our first project and make progress in my academic growth.
During my PhD study, I often felt inadequate: I wasn't particularly clever, having been just an average student during my undergraduate years; I wasn't meticulous enough, often making mistakes due to carelessness; and I wasn't always diligent, frequently being distracted.
Yet Prof. Gan never criticized me for my shortcomings. 
Instead, he consistently encouraged me, supported me, and helped me improve.
Prof. Gan also gave me considerable academic freedom, allowing me to explore different ideas and directions. 
It was through this exploration that I gradually found the direction I was willing to dedicate myself to, which has shaped who I am today.

Second, I would like to express my gratitude to several other professors who have provided invaluable guidance and support throughout my research journey. 
I am particularly thankful to Prof. Zhenli Xu from Shanghai Jiao Tong University. 
During my first semester of junior year, I had the privilege of being a visiting student in Prof. Xu's group.
Although one semester may seem brief, Prof. Xu's rigorous research approach and passion for science left a lasting impression on me. 
This period marked a significant transformation, as I evolved from a naive student into a researcher with a deep enthusiasm for scientific inquiry. Prof. Xu's comprehensive understanding of the field has been instrumental in guiding my future research direction.
I also deeply appreciate Prof. Jin-Guo Liu of Hong Kong University of Science and Technology (Guangzhou), my co-supervisor. 
Although our relationship did not begin as advisor and student, he generously shared his knowledge with me. His innovative research mindset and professional programming skills have been immensely beneficial. 
Collaborating with Prof. Liu on open-source projects and research work has been both educational and enjoyable. The coding work for this thesis would have been challenging to complete without his assistance.
I am grateful to Prof. Shidong Jiang from the Flatiron Institute, with whom I have had numerous discussions over the past year. 
Prof. Jiang's ability to address the core issues has significantly enhanced the quality of my work.
Additionally, I would like to thank Prof. Yang Xiang from Hong Kong University of Science and Technology, Prof. Pan Zhang from the Institute of Theoretical Physics, Chinese Academy of Sciences, and Prof. Feng Pan from Singapore University of Technology and Design. 
Their support has been crucial in overcoming various challenges during my thesis, enabling me to complete my doctoral research on schedule.

I would also like to thank my collaborators, including Jiuyang Liang from the Flatiron Institute, Qi Zhou from Shanghai Jiao Tong University, and Yijia Wang from the Institute of Theoretical Physics, Chinese Academy of Sciences. 
The time spent with them has always been enjoyable, and discussions with them have always brought me new knowledge and ideas. 
Most of my work has been completed in collaboration with them. 
Whether as collaborators or friends, they are impeccable. 
Now we are all ready to embark on our own paths, and I hope that in the future, we will all shine in our respective fields.
I also want to thank my friends, Zheng Yang, Tianhao Hu, Yanyu Duan, Yusheng Zhao, Zhongyi Ni, and Hongchao Li. 
Without their companionship, the life of a PhD student would have been more difficult.

Finally, I would like to thank my parents for their unwavering support and encouragement.
Over the past four years, our family has faced many challenges, but they have shouldered all the burdens alone, just to keep me from being distracted.
Their unconditional support and love are the driving forces behind my progress.

This is indeed a long journey, but it is also only the beginning of my life as a scientist.
Knowledge begins with the recognition of one's ignorance. 
The realization that the search for knowledge is unending.
More challenges await me, and I will continue to explore the world with curiosity and passion, carrying forward the spirit of scientific inquiry.


\section*{A note of publications}

Most of the material in this thesis has already appeared in the following peer-reviewed publications or articles under review:
\begin{itemize}
    \item[A.] \textbf{Xuanzhao Gao}, Qi Zhou, Zecheng Gan, Jiuyang Liang; Accurate error estimates and optimal parameter selection in Ewald summation for dielectrically confined Coulomb systems. ArXiv: 2503.18126.
    \item[B.] Zecheng Gan, \textbf{Xuanzhao Gao}, Jiuyang Liang, Zhenli Xu; Random batch Ewald method for dielectrically confined Coulomb systems. Accepted by \emph{SIAM Journal on Scientific Computing}. ArXiv: 2405.06333.
    \item[C.] Zecheng Gan, \textbf{Xuanzhao Gao}, Jiuyang Liang, Zhenli Xu; Fast algorithm for quasi-2D Coulomb systems. \emph{Journal of Computational Physics}, 524: 113733, 2025.
    \item[D.] \textbf{Xuanzhao Gao}, Zecheng Gan; Broken symmetries in quasi-2D charged systems via negative dielectric confinement. \emph{The Journal of Chemical Physics}, 161 (1): 011102, 2024.
\end{itemize}
We also present some new material in this thesis, which will appear in a forthcoming publication:
\begin{itemize}
    \item[E.] \textbf{Xuanzhao Gao}, Zecheng Gan, Yuqing Li; Efficient particle-based simulations of Coulomb systems under dielectric nanoconfinement.
\end{itemize}
In particular, the material of Chapter~\ref{chp_icmewald2d} comes from A, where we present our theoretical analysis of the Ewald splitting method for confined quasi-2D Coulomb systems.
Then in Chapters~\ref{chp_soewald2d},~\ref{chp_rbe2d}, and~\ref{chp_quasiewald}, we present our fast algorithms for quasi-2D Coulomb systems, which are contributions from B, C, and E, respectively.
In Chapter~\ref{chp_applications}, we present results on numerical experiments on confined quasi-2D Coulomb systems from B and D.
\endacknowledgments
\newpage

%%%%%%%%%%%%%%%%%%%%%%%%%%%%%%%%%%%%%%%%%%%%%%%%%%%%%%%%%%%%%%%%%%%%%%%%%
%                                                                       %
%     4) TABLE OF CONTENTS                                              %
%                                                                       %
%%%%%%%%%%%%%%%%%%%%%%%%%%%%%%%%%%%%%%%%%%%%%%%%%%%%%%%%%%%%%%%%%%%%%%%%%

\tableofcontents

%%%%%%%%%%%%%%%%%%%%%%%%%%%%%%%%%%%%%%%%%%%%%%%%%%%%%%%%%%%%%%%%%%%%%%%%%
%                                                                       %
%     5) LIST OF FIGURES (If Any)                                       %
%                                                                       %
%%%%%%%%%%%%%%%%%%%%%%%%%%%%%%%%%%%%%%%%%%%%%%%%%%%%%%%%%%%%%%%%%%%%%%%%%

\listoffigures

%%%%%%%%%%%%%%%%%%%%%%%%%%%%%%%%%%%%%%%%%%%%%%%%%%%%%%%%%%%%%%%%%%%%%%%%%
%                                                                       %
%     6) LIST OF TABLES (If Any)
%                                                                       %
%%%%%%%%%%%%%%%%%%%%%%%%%%%%%%%%%%%%%%%%%%%%%%%%%%%%%%%%%%%%%%%%%%%%%%%%%

\listoftables

%%%%%%%%%%%%%%%%%%%%%%%%%%%%%%%%%%%%%%%%%%%%%%%%%%%%%%%%%%%%%%%%%%%%%%%%%
%                                                                       %
%     7) ABSTRACT                                                       %
%                                                                       %
% \abstract and \endabstract are used to define a short Abstract for    %
% the Thesis.                                                           %
%                                                                       %
%%%%%%%%%%%%%%%%%%%%%%%%%%%%%%%%%%%%%%%%%%%%%%%%%%%%%%%%%%%%%%%%%%%%%%%%%


%%%%%%%%%%%%%%%%%%%%%%%%%%%%%%%%%%%%%%%%%%%%%%%%%%%%%%%%%%%%%%%%%%%%%%%%%
%                                                                       %
%     8) The Actual Contents                                            %
%                                                                       %
% The command \chapters MUST BE USED to ensure that the entire content  %
% of the Thesis is double-spaced (in version 1.0).                      %
%                                                                       %
% However, in version 2.0, \chapters will be automatically added in     %
% the beginning of the first chapter.                                   %
%                                                                       %
%%%%%%%%%%%%%%%%%%%%%%%%%%%%%%%%%%%%%%%%%%%%%%%%%%%%%%%%%%%%%%%%%%%%%%%%%

%%\chapters         % Not necessary with ustthesis.cls (v2.0).

%%%%%%%%%%%%%%%%%%%%%%%%%%%%%%%%%%%%%%%%%%%%%%%%%%%%%%%%%%%%%%%%%%%%%%%%%
%                                                                       %
% Each chapter is defined via the \chapter command. The usual sectional %
% commands of LaTeX are also available.                                 %
%                                                                       %
%%%%%%%%%%%%%%%%%%%%%%%%%%%%%%%%%%%%%%%%%%%%%%%%%%%%%%%%%%%%%%%%%%%%%%%%%

\chapter{Introduction}
\label{chp_intro}
\section{Background}

% Molecular dynamics (MD) simulation is one of the most powerful tools for studying the behavior of solids and fluids in a rigorous and quantitative manner and has been widely used in many areas of physics, chemistry, biology, materials science, and related disciplines.
% By tracking the positions and velocities of all particles, MD can provide a detailed description of the system's static and dynamic properties.
% In MD simulations, the system is modeled as a collection of particles, which interact with each other through potential functions and the time evolution of the system is determined by the Newton's laws of motion.
% Thus, time cost of MD simulations is dominated by the calculation of the derivatives of the potential energy with respect to the particle positions, i.e. the force between particles.

% The potential can be generally classified into two categories: short-range and long-range.
% For the short-range potential such as the Lennard-Jones potential, the computational cost scales as $\mathcal O(N)$ with the number of particles $N$ based on the neighbor list algorithm and the real space truncation.
% For the long-range potential such as the Coulomb potential, the computational cost scales as $\mathcal O(N^2)$ in the naive implementation, which is prohibitive for large-scale systems.
% Fast algorithms are thus highly desirable for simulating the Coulomb systems.
% For isotropic Coulomb systems, deterministic algorithms with complexity of $\mathcal O(N)$ or $\mathcal O(N\log N)$ have been developed, which usually fall into one of the two categories: fast multipole methods (FMM)~\cite{greengard1987fast,cheng1999fast,ying2004kernel} and fast Fourier transform (FFT) based Ewald-splitting methods~\cite{hockney2021computer,darden1993particle,essmann1995smooth}.

Quasi two dimensional (quasi-2D) Coulomb systems~\cite{mazars2011long}, which are macroscopic in two dimensions but with atomic-size thickness in the other, have caught much attention in studies of magnetic and liquid crystal films, super-capacitors, crystal phase transitions, dusty plasmas, ion channels, superconductive materials and quantum devices~\cite{kawamoto2002history, hille2001ionic, teng2003microscopic, Messina2017PRL, mazars2011long, saito2016highly, liu20192d}. 
Typically, such systems possess a nano-sized longitudinal thickness in the $z$ direction, achieved through confinement, bulk-like and modeled as periodic in the transverse $xy$ directions, hence endowed with an  inherent multi-scale nature.
Due to the confinement effect, such systems can exhibit various interesting behaviors for future nanotechnologies;
prototype examples include graphene~\cite{novoselov2004electric}, metal dichalcogenide monolayers~\cite{kumar2012tunable}, and colloidal monolayers~\cite{mangold2003phase}.


To study the behavior of quasi-2D Coulomb systems, molecular dynamics (MD) simulations are one of the most powerful tools.
By tracking the positions and velocities of all particles, MD can provide a detailed description of the system's static and dynamic properties.
In MD simulations, the system is modeled as a collection of particles, which interact with each other through potential functions and the time evolution of the system is determined by the Newton's laws of motion.
Thus, time cost of MD simulations is dominated by the calculation of the derivatives of the potential energy with respect to the particle positions, i.e. the force between particles.
For the short-range potential such as the Lennard-Jones potential, the computational cost scales as $\mathcal O(N)$ with the number of particles $N$ based on the neighbor list algorithm and the real space truncation.
For long-range interactions such as the Coulomb interaction, the potential energy can be generally expressed as the lattice kernel summations in the form of
\begin{equation}\label{eq::phi12}
	\phi(\V x) = \sum_{\V{m}} \sum_{j = 1}^{N} \rho_j K(\V{x} - \V{y}_j + \V{m} \circ \V{L})\;,
\end{equation}
where~$\V{x}, \V{y}_j \in \mathbb{R}^d$ are $d$-dimensional vectors in a rectangular box $\Omega$ with $\V{L}$ the vector of its edge lengths, $\rho_j$ refers to the density or weight, $\V{m} \in \mathbb{Z}^{d^\prime} \otimes \{0\}^{d-d^\prime}$ exerts periodicity in the first $d^{\prime}$ directions (with $d^{\prime}\leq d$), ``$\circ$'' represents the Hadamard product, and~$K(\V{x})$ is the kernel function whose form depends on the interested physical problem. 
If~$d^\prime = d$, the system is called fully-periodic, $d^\prime=0$, it is in a free-space, otherwise it is called partially-periodic. 
For such long-range interactions, the computational cost scales as $\mathcal O(N^2)$ in the naive pair-wise implementation, which greatly limits the size of the system that can be studied.


\section{Significance}

To reduce the computational cost of evaluating the kernel summations in Eq.~\eqref{eq::phi12}, various algorithms have been developed for fully-periodic or free-space Coulomb systems, such as the fast multipole methods (FMM)~\cite{greengard1987fast,cheng1999fast,ying2004kernel}, fast Fourier transform (FFT) based Ewald-splitting methods~\cite{hockney2021computer,darden1993particle,essmann1995smooth} and the recently proposed random batch Ewald (RBE) method~\cite{jin2021random, liang2022superscalability,liang2024JCP}.
These fast algorithms have been successfully applied to large-scale simulations of Coulomb systems under various ensembles and reaching complexity of $\mathcal O(N\log N)$ or even $\mathcal O(N)$.

However, different from the fully-periodic or free-space systems, quasi-2D systems possess a reduced symmetry, which gives rise to new phenomena, but also brings formidable challenges in both theory and computation.
The first challenge comes from the involved \emph{long-range} interaction kernels, including but not limited to Coulomb and dipolar kernels in electrostatics, Oseen and Rotne-Prager-Yamakawa kernels in hydrodynamics and the static exchange-correlation kernels in density functional theory calculations. 
% For fully-periodic or free-space systems, $\mathcal O(N)$ fast algorithms have been developed; but the field is still under developing for partially-periodic systems.
The anisotropy of such systems poses extra challenges for simulations:
(1) the periodic and non-periodic directions need to be handled separately due to their different boundary conditions and length scales;
(2) the convergence properties of the lattice kernel summation requires careful consideration, which largely depend on the well-poseness of the underlying PDEs. 
Another challenge comes from practical applications.
To accurately determine the phase diagram of a many-body system may require thousands of simulation runs under different conditions~\cite{levin2002electrostatic}, each with billions of time steps to sample ensemble averages.
Moreover, to eliminate the finite size effect, millions of free particles need to be simulated. Such large-scale simulations are especially required for quasi-2D systems, so as to accommodate its strong anisotropy, and resolving possible boundary layers forming near the confinement surfaces~\cite{mazars2011long}. 
The cumulative impact of these considerations poses significant challenges for numerical simulations for quasi-2D systems. 

To address these issues associated with the particle-based simulation of quasi-2D systems, a variety of numerical methods have been developed.
Most of them also fall into two categories: 
(1) Fourier spectral methods~\cite{lindbo2012fast,nestler2015fast,doi:10.1021/acs.jctc.3c01124, maxian2021fast}, where particles are first smeared onto grids, and subsequently the underlying PDE is solved in Fourier domain where fast Fourier transform (FFT) can be used for acceleration; 
(2) adaptive tree-based methods, where fast multipole method (FMM)~\cite{greengard1987fast} or tree code~\cite{Barnes1986Nature} orginally proposed for free-space systems can be extended to quasi-2D systems by careful extension to match the partially-periodic boundary conditions~\cite{yan2018flexibly,liang2020harmonic}. 
Alternative methods have also been proposed, such as the Lekner summation-based MMM2D method~\cite{arnold2002novel}, multilevel summation methods~\cite{doi:10.1021/ct5009075,greengard2023dual}, and correction-based approaches such as Ewald3DC~\cite{yeh1999ewald} and EwaldELC~\cite{arnold2002electrostatics}, which first solve a fully-periodic system and then add the partially-periodic correction terms. 
By combining with either FFT or FMM, these methods achieve $\mathcal{O}(N\log N)$ or even $\mathcal{O}(N)$ complexity. 

However, the issue of large-scale simulation of quasi-2D systems is still far from settled.
A few challenges remains. First, FFT-based methods need extra techniques to properly handle the non-periodic direction, such as truncation~\cite{parry1975electrostatic}, regularization~\cite{nestler2015fast}, or periodic extension~\cite{lindbo2012fast}, which may lead to algebraic convergence or require extra zero-padding to guarantee accuracy. 
Recent advancements by Shamshirgar \emph{et al.}~\cite{shamshirgar2021fast}, combining spectral solvers with kernel truncation methods (TKM)~\cite{vico2016fast}, have reduced the zero-padding factor from $6$ to $2$~\cite{lindbo2012fast}, which still requires doubling the number of grids with zero-padding. 
% A similar reduction is also reported in the work of Maxian \emph{et al. }\cite{maxian2021fast}. 
Second, the periodization of FMM needs to encompass more near-field contributions from surrounding cells~\cite{yan2018flexibly,barnett2018unified}. The recently proposed 2D-periodic FMM~\cite{PEI2023111792} may offer a promising avenue; however, it has not yet been extended to partially-periodic problems.
Finally, it is worth noting that most of the aforementioned issues will become more serious when $L_z\ll \min\{L_x, L_y\}$, in which case the Ewald series summation will converge much slower~\cite{arnold2002electrostatics}, and the zero-padding issue of FFT-based methods also becomes worse~\cite{maxian2021fast}.
Thus, efficient and accurate simulation of quasi-2D systems is still an open problem and a great challenge.

In this thesis, a new class of fast summation methods for quasi-2D Coulomb systems are introduced, which combines the Ewald splitting technique and the random batch sampling method.
These methods are able to achieve $\mathcal{O}(N\log N)$ complexity and are able to handle strongly confined quasi-2D systems with large aspect ratios and sharp dielectric interfaces.
Rigorous error estimates and complexity analysis are provided, further validated by numerical tests.

\section{Thesis Outlines}

The content of this thesis is organized as follows.

In Chapter~\ref{chp_preliminaries}, we introduce the physical model and mathematical notations for dielectric confined quasi-2D Coulomb systems, and provide a concise overview of the Ewald summation formula for fully-periodic and doubly-periodic Coulomb systems.
We will also revisit the random batch sampling method for fully-periodic Coulomb systems, which is the basis of the methods proposed in this thesis.

In Chapter~\ref{chp_soewald2d}, we first introduce the kernel approximation technique called the sum-of-exponential (SOE) approximation, which approximates an arbitrary kernel function by a linear combination of exponential functions.
Then, based on the SOE approximation, we propose an algorithm named the sum-of-exponential Ewald2D method, which utilizes the SOE approximation and the random batch sampling method to accelerate the Ewald2D summation for quasi-2D Coulomb systems without dielectric confinements.
Numerical results are provided to validate the accuracy and efficiency of the proposed method.

In Chapter~\ref{chp_rbe2d}, we first introduce the image charge method, which is a popular approach for treating the dielectric confinements in quasi-2D systems.
Additionally, a rigorous error analysis will be provided for the combination of the image charge method and the Ewald summation formula.
Then based on the error analysis, we propose a new algorithm named the random batch Ewald2D method, which is able to handle the quasi-2D Coulomb systems with dielectric confinements and is scalable to large-scale simulations on modern supercomputers.


In Chapter~\ref{chp_quasiewald}, we focus on the so-called negatively dielectirc confined quasi-2D systems, where the dielectric constant of the confinement can be negative, which leads to divergence in the previous methods.
We propose a new algorithm named the quasi-Ewald method, where we introduce a new type of splitting technique, called the quasi-Ewald splitting, to handle the divergence issue.
With the proposed method, we observe spontaneous symmetry breaking (SSB) phenomena in symmetrically charged binary particle systems under such negative dielectric confined systems.

In Chapter~\ref{chp_conclusion}, we provide a conclusion for this thesis, where we summarize the advantages and disadvantages of the proposed methods, and discuss the potential applications of the proposed methods as well as the future work.
\newpage

\chapter{The Sum-of-Exponentials Ewald2D Method}
\label{chp_soewald2d}
\section{The Sum-of-Exponentials Approximation}

\section{The Algorithm}

\section{Numerical Results}
\newpage

\chapter{The Random Batch Ewald2D Method}
\label{chp_rbe2d}
\section{The Image Charge Method and its Error Estimation}

\section{The Algorithm}

\section{Numerical Results}


\newpage

\chapter{The Quasi-Ewald Method}
\label{chp_quasiewald}
\section{The Algorithm}

\section{Numerical Results}

\section{Broken Symmetry in Quasi-2D Coulomb Systems}
\newpage

\chapter{Conclusions and Future Work}
\label{chp_conclusion}
This is a conclusion.
\newpage

%%%%%%%%%%%%%%%%%%%%%%%%%%%%%%%%%%%%%%%%%%%%%%%%%%%%%%%%%%%%%%%%%%%%%%%%%
%                                                                       %
%      9) BIBLIOGRAPHY                                                  %
%                                                                       %
% This example uses bibtex to generate the required Bibliography. Refer %
% to the % the file ustthesis_test.bib for the entries of the           %
% Bibliography. Note that only the cited entries are printed.           %
%                                                                       %
% If BibTeX is not used to typeset the bibliography, replace the        %
% following line with the \begin{thebibliography} and \end{bibliography}%
% commands (the "thebibliography" environment) to process the           %
% Bibliography.                                                         %
%                                                                       %
%%%%%%%%%%%%%%%%%%%%%%%%%%%%%%%%%%%%%%%%%%%%%%%%%%%%%%%%%%%%%%%%%%%%%%%%%

%%%%%%%%%%%%%%%%%%%%%%%%%%%%%%%%%%%%%%%%%%%%%%%%%%%%%%%%%%%%%%%%%%%%%%%%%
%                                                                       %
% The recommended bibliography style is the IEEE bibliography style.    %
% "ustbib" defines the IEEE bibliography standard with the added        %
% ability of sorting the items by name of author.                       %
%                                                                       %
% If you are not using BibTeX to process your Bibliography, comment out %
% the following line.                                                   %
%                                                                       %
%%%%%%%%%%%%%%%%%%%%%%%%%%%%%%%%%%%%%%%%%%%%%%%%%%%%%%%%%%%%%%%%%%%%%%%%%

\addcontentsline{toc}{chapter}{Bibliography}
\bibliographystyle{IEEEtran}
\bibliography{references}
\newpage

%%%%%%%%%%%%%%%%%%%%%%%%%%%%%%%%%%%%%%%%%%%%%%%%%%%%%%%%%%%%%%%%%%%%%%%%%
%                                                                       %
%     10) APPENDIX (If Any)                                              %
%                                                                       %
% \appendix command marks the beginning of the APPENDIX part of the     %
% Thesis. The usual \chapter command is used for the different chapters %
% of the Appendix.                                                      %
%                                                                       %
%%%%%%%%%%%%%%%%%%%%%%%%%%%%%%%%%%%%%%%%%%%%%%%%%%%%%%%%%%%%%%%%%%%%%%%%%
\appendix
\chapter{Tips on HKUST thesis preparation}
\label{chp_tips}
% \section{Sideway page}
Here is an example to have a page turned side way in the PDF. It is useful for having wide figures and tables in the thesis. 

\begin{landscape} 
\begin{table}
\caption{Pearson's
        $\rho $        correlation of participated metrics with the WMT 2016 official average
        direct assessment human judgments on newstest 10k hybrid super-sampled
        systems at system level. Correlations of metrics not significantly
        outperformed by any other for that language pair are highlighted in
        bold.}
\centering

\begin{tabular}{m{0.36\textwidth}cccccccc|cccccccc}
\hline & \multicolumn{8}{c|}{\textbf{into-English}} & \multicolumn{8}{c}{\textbf{out-of-English}} \\\textbf{Metric} & \textbf{cs} & \textbf{de} & \textbf{fi} & \textbf{lv} & \textbf{ru} & \textbf{tr} & \textbf{zh} & \textbf{avg.} & \textbf{cs} & \textbf{de} & \textbf{fi} & \textbf{lv} & \textbf{ru} & \textbf{tr} & \textbf{zh} & \textbf{avg.} \\ \hline
Blend  & .963 & \textbf{.969} & .956 & .976 & \textbf{.957} & .981 & .890 & .956 & -- & -- & -- & -- & .950 & -- & -- & -- \\ \hline
BEER & .966 & .952 & .954 & .974 & .930 & .969 & .897 & .949 & .963 & .829 & .975 & .923 & .942 & .968 & .906 & .929 \\ \hline
UHH\_TSKM & .990 & .929 & .918 & \textbf{.986} & .908 & .982  & .896  & .944  & --  & --  & --  & --  & --  & --  & --  & --  \\ \hline
CDER & .983  & .922  & .925  & .981  & .916  & .970  & \textbf{.898 } & .942  & .958  & .803  & .962  & .911  & .922  & .948  & \textbf{.975 } & .926  \\ \hline
TreeAggreg & .977  & .913  & \textbf{.975} & .983  & .912  & \textbf{.983 } & .854  & .942  & .942  & .765  & .963  & .915  & .919  & .971  & .933  & .915  \\ \hline
chrF++ & .935  & .957  & .924  & .970  & .938  & .957  & .876  & .937  & .966  & .835  & .977  & .944  & .942  & .975  & .968  & .944  \\ \hline
chrF & .933  & .960  & .935  & .965  & .946  & .941  & .855  & .934  & .968  & .845  & \textbf{.979 } & .945  & .947  & \textbf{.980 } & .969  & .947  \\ \hline
NIST & \textbf{.994} & .917  & .928  & .957  & .904  & .969  & .831  & .929  & .954  & .761  & .957  & .914  & .917  & .976  & .968  & .921  \\ \hline
TER & .983  & .899  & .950  & .967  & .905  & .951  & .837  & .927  & .951  & .790  & .959 & .888 & .930 & .958 & .965 & .920 \\ \hline
BLEU & .964  & .914  & .906  & .974  & .907  & .969  & .852  & .927  & .945  & .793  & .919  & .839  & .893  & .916  & .969  & .896  \\ \hline
MEANT-dvw.cos.wmax.n2.m.r8.$\alpha $1.0.$\beta $0.1 & .921  & .942  & .939  & .967  & .955  & .931  & .836  & .927  & --  & .844  & --  & --  & --  & --  & .944  & --  \\ \hline
CharacTER & .963  & .965  & .944  & .927  & .948  & .946  & .740  & .919  & \textbf{.973 } & \textbf{.893 } & .970  & .892  & .929  & .961  & .914  & .933 \\ \hline
WER & .981  & .889  & .946  & .965  & .900  & .922  & .828  & .919  & .949  & .797  & .959  & .884  & .931  & .947  & .951  & .917  \\ \hline
PER & .967  & .920  & .892  & .958  & .904  & .898  & .866  & .915  & .960  & .680  & .939  & .817  & .876  & .955  & .893  & .874  \\ \hline
MEANT-dvw.cos.wmax.n2.m.r8.$\alpha $1.0.$\beta $0.0 & .896  & .928  & .931  & .960  & .952  & .895  & .799  & .909  & .968  & .753  & .970  & \textbf{.947 } & \textbf{.955 } & .980  & .931  & .929  \\ \hline
AutoDA & .440  & .951  & .922  & .970  & .902  & .914  & .734  & .833  & .967  & .602  & .879  & .731  & .850  & .586  & .968  & .797  \\ \hline
\end{tabular}
\label{tbl_wmt17-sys}
\end{table}
\end{landscape} 


%%%%%%%%%%%%%%%%%%%%%%%%%%%%%%%%%%%%%%%%%%%%%%%%%%%%%%%%%%%%%%%%%%%%%%%%%
%                                                                       %
%     11) BIOGRAPHY (optional)                                          %
%                                                                       %
% \biography and \endbiography are used to define the optional          %
% Biography of the author of the Thesis.                                %
%                                                                       %
%%%%%%%%%%%%%%%%%%%%%%%%%%%%%%%%%%%%%%%%%%%%%%%%%%%%%%%%%%%%%%%%%%%%%%%%%

\end{document}
