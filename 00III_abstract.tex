This thesis focuses on the quasi-two-dimensional (quasi-2D) Coulomb systems, which are prevalent in fields such as materials science, biophysics, and electrochemistry. 
These systems, characterized by their macroscopic dimensions in two directions and atomic-scale thickness in the third, present significant computational challenges due to their complex boundary conditions and multi-scale nature, especially when confined by dielectric interfaces.
However, traditional simulation methods face significant challenges in these systems due to the reduced symmetry and the long-range nature of Coulomb interactions.

The research begins with a comprehensive theoretical analysis of Ewald summation methods tailored for quasi-2D systems, providing rigorous error estimates and optimal parameter selection strategies. 
Building on this foundation, three innovative algorithms are developed: the sum-of-exponentials Ewald2D (SOEwald2D) method, the random batch Ewald2D (RBE2D) method, and the quasi-Ewald Method (QEM), which are designed for simulating quasi-2D Coulomb systems in homogeneous dielectric environments, dielectrically confined quasi-2D systems, and systems with negative dielectric confinement, respectively.
These algorithms integrate Ewald splitting techniques with random batch sampling, achieving optimal $O(N)$ computational complexity while maintaining high accuracy. 
% Notably, these methods are mesh-free and demonstrate superior parallel scalability compared to traditional FFT-based approaches.
The thesis validates these methods through extensive numerical experiments, and the results highlight the methods' ability to accurately capture both structural and dynamical properties while significantly outperforming existing approaches in computational efficiency and scalability, particularly for strongly confined systems.

This thesis demonstrates practical applications of the proposed methods through comprehensive studies of confined quasi-2D Coulomb systems. 
These include all-atom molecular dynamics simulations of confined SPC/E water models and detailed numerical investigations of negatively confined electrolyte systems. 
Notably, the research reveals a previously undiscovered phenomenon where symmetrically charged binary particle systems exhibit broken symmetries solely due to confinement effects.

The advancements presented in this thesis address long-standing computational challenges in simulating quasi-2D Coulomb systems, offering powerful tools for studying confined charged systems across scales. 
The theoretical insights and algorithmic innovations open new avenues for exploring complex phenomena in biological systems, materials science, and electrochemistry.