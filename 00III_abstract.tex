This thesis develops novel fast algorithms for simulating quasi-two-dimensional (quasi-2D) Coulomb systems, which are characterized by double periodicity in two dimensions and confinement in the third dimension. Such systems are crucial for studying various physical phenomena, including magnetic films, super-capacitors, ion channels, and quantum devices. Despite their importance, efficient simulation of quasi-2D systems remains challenging due to their reduced symmetry, multi-scale nature, and the need to handle dielectric interfaces.

We introduce a new class of fast summation methods that synergistically combines the Ewald splitting technique with random batch sampling. Our approach achieves optimal $\mathcal{O}(N)$ computational complexity while maintaining high efficiency for strongly confined quasi-2D systems, even in challenging scenarios involving large aspect ratios and sharp dielectric interfaces. The thesis makes three main methodological contributions:

First, we develop the sum-of-exponential Ewald2D (SOEwald2D) method, which utilizes sum-of-exponential approximation and random batch sampling to accelerate the Ewald2D summation for quasi-2D Coulomb systems without dielectric confinements. Second, we propose the random batch Ewald2D (RBE2D) method for systems with dielectric confinements, providing rigorous error analysis for the combination of image charge method and Ewald summation. Third, we introduce the quasi-Ewald method (QEM) to handle systems with negative dielectric confinements, where traditional methods break down due to unusual polarization effects.

For each method, we provide comprehensive theoretical analysis including rigorous error estimates and complexity bounds. These theoretical predictions are thoroughly validated through extensive numerical experiments. Our methods demonstrate excellent scalability for large-scale simulations on modern supercomputing architectures. Using these new algorithms, we discover remarkable spontaneous symmetry breaking phenomena emerging in symmetrically charged binary particle systems under negative dielectric confinement.

This work advances the field of computational physics by providing efficient and accurate tools for studying quasi-2D Coulomb systems, opening new possibilities for investigating complex physical phenomena at interfaces and in confined geometries. The methodological framework developed here can be extended to other partially periodic systems and has broader implications for simulating various long-range interaction problems.