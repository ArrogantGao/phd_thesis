This thesis focuses on the confined quasi-two-dimensional (quasi-2D) Coulomb systems, which are prevalent in fields such as materials science, biophysics, and electrochemistry. 
These systems, characterized by their macroscopic dimensions in two directions and atomic-scale thickness in the third, present significant computational challenges for traditional simulation methods due to their reduced symmetry, complex boundary conditions and multi-scale nature.

The research begins with a comprehensive theoretical analysis of Ewald summation methods tailored for quasi-2D systems, providing rigorous error estimates and optimal parameter selection strategies. 
Then, three innovative algorithms are developed: the sum-of-exponentials Ewald2D (SOEwald2D) method, the random batch Ewald2D (RBE2D) method, and the quasi-Ewald Method (QEM), which are designed for simulating quasi-2D Coulomb systems under homogeneous medium, dielectrically confined quasi-2D systems, and systems confined by meta-materials characterized by negative dielectric constant, respectively.
These algorithms combine Ewald splitting techniques with novel approaches such as random batch sampling, kernel approximation, and singularity subtraction, achieving optimal $O(N)$ computational complexity while maintaining high accuracy even for strongly confined systems.
The thesis validates these methods through extensive numerical experiments, and the results highlight the methods' ability to accurately capture both structural and dynamical properties while outperforming existing approaches.

% This thesis demonstrates practical applications of the proposed methods through comprehensive studies of confined quasi-2D Coulomb systems. 
% These include all-atom molecular dynamics simulations of confined SPC/E water models and detailed numerical investigations of negatively confined electrolyte systems. 
% Notably, the research reveals a previously undiscovered phenomenon where symmetrically charged binary particle systems exhibit broken symmetries solely due to confinement effects.

\rev{
    This thesis also presents new physical insights into the behavior of confined Coulomb systems.
    It reveals a previously undiscovered phenomenon where symmetrically charged binary particle systems exhibit broken symmetries solely due to confinement effects.
    This finding has important implications for understanding the behavior of confined Coulomb systems and for designing new materials with tailored properties.
}

The advancements presented in this thesis address long-standing computational challenges in simulating quasi-2D Coulomb systems, offering powerful tools for studying confined charged systems across scales. 
The theoretical insights and algorithmic innovations open new avenues for exploring complex phenomena in biological systems, materials science, and electrochemistry.