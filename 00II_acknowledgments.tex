Thanks your supervisor(s) and the defense committee. 

Thanks everyone help in your thesis research.

Thanks everyone support you throughout the way.

Thanks your loved ones.


\section*{A note of publications}

Most of the material in this thesis has already appeared in the following peer-reviewed publications or articles under review:
\begin{itemize}
    \item[A.] Xuanzhao Gao, Qi Zhou, Zecheng Gan, Jiuyang Liang; Accurate error estimates and optimal parameter selection in Ewald summation for dielectrically confined Coulomb systems. ArXiv: 2503.18126.
    \item[B.] Zecheng Gan, Xuanzhao Gao, Jiuyang Liang, Zhenli Xu; Random batch Ewald method for dielectrically confined Coulomb systems. Accepted by \emph{SIAM Journal on Scientific Computing}. ArXiv: 2405.06333.
    \item[C.] Zecheng Gan, Xuanzhao Gao, Jiuyang Liang, Zhenli Xu; Fast algorithm for quasi-2D Coulomb systems. \emph{Journal of Computational Physics}, 524: 113733, 2025.
    \item[D.] Xuanzhao Gao, Zecheng Gan; Broken symmetries in quasi-2D charged systems via negative dielectric confinement. \emph{The Journal of Chemical Physics}, 161 (1): 011102, 2024.
\end{itemize}
We also present some new material in this thesis, which will appear in a forthcoming publication:
\begin{itemize}
    \item[E.] Xuanzhao Gao, Zecheng Gan, Yuqing Li; Efficient particle-based simulations of Coulomb systems under dielectric nanoconfinement.
\end{itemize}
In particular, the material of Chapter~\ref{chp_icmewald2d} comes from A, where we present our theoretical analysis of the Ewald splitting method for confined quasi-2D Coulomb systems.
Then in Chapters~\ref{chp_soewald2d},~\ref{chp_rbe2d}, and~\ref{chp_quasiewald}, we present our fast algorithms for quasi-2D Coulomb systems, which are contributions from B, C, and E, respectively.
In Chapter~\ref{chp_applications}, we present results on numerical experiments on confined quasi-2D Coulomb systems from B, C, D, and E.