% acknowledgments

Completing this thesis is a long journey, and I would like to thank everyone who has helped me along the way.

First and foremost, I would like to express my deepest gratitude to my advisor, Prof. Zecheng Gan from Hong Kong University of Science and Technology (Guangzhou), under whose guidance the main work of this thesis was completed.
In the summer of 2021, I became Prof. Gan's first PhD student, marking the beginning of our journey together.
During my undergraduate studies, I majored in Applied Physics. 
Although I chose to enter this entirely new field due to my interest in applied mathematics and scientific computing, I had not received sufficient training in these areas.
However, Prof. Gan never dismissed me for my lack of knowledge. Instead, he patiently tutored me in the fundamentals of scientific computing and applied mathematics, and provided guidance and support whenever I encountered difficulties.
Under these circumstances, I was able to quickly begin our first project and make progress in my academic growth.
During my PhD study, I often felt inadequate: I wasn't particularly clever, having been just an average student during my undergraduate years; I wasn't meticulous enough, often making mistakes due to carelessness; and I wasn't always diligent, frequently being distracted.
Yet Prof. Gan never criticized me for my shortcomings. 
Instead, he consistently encouraged me, supported me, and helped me improve.
Prof. Gan also gave me considerable academic freedom, allowing me to explore different ideas and directions. 
It was through this exploration that I gradually found the direction I was willing to dedicate myself to, which has shaped who I am today.

Second, I would like to express my gratitude to several other professors who have provided invaluable guidance and support throughout my research journey. 
I am particularly thankful to Prof. Zhenli Xu from Shanghai Jiao Tong University. 
During my first semester of junior year, I had the privilege of being a visiting student in Prof. Xu's group.
Although one semester may seem brief, Prof. Xu's rigorous research approach and passion for science left a lasting impression on me. 
This period marked a significant transformation, as I evolved from a naive student into a researcher with a deep enthusiasm for scientific inquiry. Prof. Xu's comprehensive understanding of the field has been instrumental in guiding my future research direction.
I am also deeply appreciative of Prof. Jin-Guo Liu from Hong Kong University of Science and Technology (Guangzhou), my co-supervisor. 
Although our relationship did not begin as advisor and student, he generously shared his knowledge with me. His innovative research mindset and professional programming skills have been immensely beneficial. 
Collaborating with Prof. Liu on open-source projects and research work has been both educational and enjoyable. The coding work for this thesis would have been challenging to complete without his assistance.
I am grateful to Prof. Shidong Jiang from the Flatiron Institute, with whom I have had numerous discussions over the past year. 
Prof. Jiang's ability to address the core issues has significantly enhanced the quality of my work.
Additionally, I would like to thank Prof. Yang Xiang from Hong Kong University of Science and Technology, Prof. Pan Zhang from the Institute of Theoretical Physics, Chinese Academy of Sciences, and Prof. Feng Pan from Singapore University of Technology and Design. 
Their support has been crucial in overcoming various challenges during my thesis, enabling me to complete my doctoral research on schedule.

I would also like to thank my collaborators, including Jiuyang Liang from the Flatiron Institute, Qi Zhou from Shanghai Jiao Tong University, and Yijia Wang from the Institute of Theoretical Physics, Chinese Academy of Sciences. 
The time spent with them has always been enjoyable, and discussions with them have always brought me new knowledge and ideas. 
Most of my work has been completed in collaboration with them. 
Whether as collaborators or friends, they are impeccable. 
Now we are all ready to embark on our own paths, and I hope that in the future, we will all shine in our respective fields.
I also want to thank my friends, Zheng Yang, Tianhao Hu, Yanyu Duan, Yusheng Zhao, Zhongyi Ni, and Hongchao Li. 
Without their companionship, the life of a PhD student would have been more difficult.

Finally, I would like to thank my parents for their unwavering support and encouragement.
Over the past four years, our family has faced many challenges, but they have shouldered all the burdens alone, just to keep me from being distracted.
Their unconditional support and love are the driving forces behind my progress.

This is indeed a long journey, but it is also only the beginning of my life as a scientist.
Knowledge begins with the recognition of one's ignorance. 
The realization that the search for knowledge is unending.
More challenges await me, and I will continue to explore the world with curiosity and passion, carrying forward the spirit of scientific inquiry.


\section*{A note of publications}

Most of the material in this thesis has already appeared in the following peer-reviewed publications or articles under review:
\begin{itemize}
    \item[A.] \textbf{Xuanzhao Gao}, Qi Zhou, Zecheng Gan, Jiuyang Liang; Accurate error estimates and optimal parameter selection in Ewald summation for dielectrically confined Coulomb systems. ArXiv: 2503.18126.
    \item[B.] Zecheng Gan, \textbf{Xuanzhao Gao}, Jiuyang Liang, Zhenli Xu; Random batch Ewald method for dielectrically confined Coulomb systems. Accepted by \emph{SIAM Journal on Scientific Computing}. ArXiv: 2405.06333.
    \item[C.] Zecheng Gan, \textbf{Xuanzhao Gao}, Jiuyang Liang, Zhenli Xu; Fast algorithm for quasi-2D Coulomb systems. \emph{Journal of Computational Physics}, 524: 113733, 2025.
    \item[D.] \textbf{Xuanzhao Gao}, Zecheng Gan; Broken symmetries in quasi-2D charged systems via negative dielectric confinement. \emph{The Journal of Chemical Physics}, 161 (1): 011102, 2024.
\end{itemize}
We also present some new material in this thesis, which will appear in a forthcoming publication:
\begin{itemize}
    \item[E.] \textbf{Xuanzhao Gao}, Zecheng Gan, Yuqing Li; Efficient particle-based simulations of Coulomb systems under dielectric nanoconfinement.
\end{itemize}
In particular, the material of Chapter~\ref{chp_icmewald2d} comes from A, where we present our theoretical analysis of the Ewald splitting method for confined quasi-2D Coulomb systems.
Then in Chapters~\ref{chp_soewald2d},~\ref{chp_rbe2d}, and~\ref{chp_quasiewald}, we present our fast algorithms for quasi-2D Coulomb systems, which are contributions from B, C, and E, respectively.
In Chapter~\ref{chp_applications}, we present results on numerical experiments on confined quasi-2D Coulomb systems from B and D.