\section{Force expressions of SOEwald2D} \label{app::force}

The Fourier component of force acting on the $i$th particle can be evaluated by taking the gradient of the energy with respect to the particle's position vector $\bm{r}_{i}$,
\begin{equation}\label{eq::Fi}
	\V{F}_{\ell}^i  \approx \V{F}^{i}_{\text{l},\text{SOE}} = -\grad_{\V{r}_{i}} U_{\ell,\text{SOE}} =  -\sum_{\bm{h}\neq \bm{0}} \grad_{\V{r}_{i}}U_{\ell,\text{SOE}}^{\V{h}} -\grad_{\V{r}_{i}} U_{\ell,\text{SOE}}^{\V{0}}
\end{equation}
where
\begin{align}    
	\grad_{\V{r}_{i}}U_{\ell,\text{SOE}}^{\V{h}} &= - \frac{\pi q_{i}}{L_x L_y} \left[ \sum_{1 \leq j < i} q_{j} \grad_{\V{r}_{i}} \varphi_{\text{SOE}}^{\V{h}}(\V{r}_{i}, \V{r}_{j}) +  \sum_{i < j \leq N} q_{j} \grad_{\V{r}_{i}} \varphi_{\text{SOE}}^{\V{h}}(\V{r}_{j}, \V{r}_{i}) \right]\;,\\
	\grad_{\V{r}_{i}} U_{\ell,\text{SOE}}^{\V{0}} &= - \frac{2 \pi q_{i}}{L_x L_y} \left[ \sum_{1 \leq j < i} q_{j} \grad_{\V{r}_{i}} \varphi^{\bm{0}}_{\text{SOE}}(\bm{r}_{i}, \bm{r}_{j}) + \sum_{i < j \leq N} q_{j} \grad_{\V{r}_{i}} \varphi^{\bm{0}}_{\text{SOE}}(\bm{r}_{j}, \bm{r}_{i})\right]\;.
\end{align}
Using the approximation Eqs.~\eqref{eq::SOEphi},~\eqref{eq::dz_plus} and~\eqref{eq::dz_minus}, one can write the derivative in periodic directions as
\begin{equation}
	\begin{split}
		\partial_{\bm{\rho}_{i}} \varphi_{\text{SOE}}^{\V{h}}(\V{r}_{i}, \V{r}_{j}) & = \frac{\m{i} \V{h} e^{ \m{i} \V{h} \cdot \V{\rho}_{ij}}}{h} \left[\xi^{+}_M(h, z_{ij})+\xi^{-}_M(h, z_{ij})\right]\\
		&=\frac{2\alpha e^{-h^2/(4\alpha^2)}}{\sqrt{\pi}h} \m{i} \bm{h}e^{\m{i} \V{h} \cdot \V{\rho}_{ij}} \sum_{\ell = 1}^{M}  \frac{w_l}{\alpha^2 s_l^2 - h^2}\left( 2 \alpha s_l e^{-h z_{ij}} - 2 h e^{-\alpha s_l z_{ij}}\right),
	\end{split}
\end{equation}
and in~$z$ direction as
\begin{equation}\label{eq:z-der}
	\begin{split}
		\partial_{z_{i}} \varphi_{\text{SOE}}^{\V{h}}(\V{r}_{i}, \V{r}_{j}) & = \frac{e^{\m{i} \V{h} \cdot \V{\rho}_{ij}}}{h} \left[\partial_{z_{i}} \xi^{+}_M(h, z_{ij}) + \partial_{z_{i}} \xi^{-}_M(h, z_{ij})\right]\\
		&=\frac{2 \alpha e^{-h^2/(4\alpha^2)}}{\sqrt{\pi}}e^{ \m{i} \V{h} \cdot \V{\rho}_{ij}} \sum_{\ell = 1}^{M}  \frac{w_l}{\alpha^2 s_l^2 - h^2}\left( - 2 \alpha s_l e^{-h z_{ij}} + 2 \alpha s_l e^{- \alpha s_l z_{ij}}\right)\;.
	\end{split}
\end{equation}
The partial derivatives of zero-frequency mode with respect to the periodic directions are zero, and the SOE approximation of its $z$-derivative is given by
\begin{equation}\label{eq:dzphi_0}
	\begin{split}
		\partial_{z_{i}} \varphi^{\bm{0}}_{\text{SOE}}(\bm{r}_{i},\bm{r}_{j}) 
		& = \sum_{l=1}^{M} \frac{w_l}{\sqrt{\pi}} \partial_{z_{i}} \left[\frac{2z_{ij}}{s_l}+\left(\frac{1}{\alpha} - \frac{2z_{ij}}{s_l}\right)e^{-\alpha s_l z_{ij}}\right] \\
		& = \sum_{l=1}^{M} \frac{w_l}{\sqrt{\pi}} \left[ \frac{2}{s_l} - \left( s_l + \frac{2}{s_l} - 2 \alpha z_{ij} \right) e^{-\alpha s_l z_{ij}} \right]\;.
	\end{split}
\end{equation}

It is important to note that the computation of Fourier space forces using Eq.~\eqref{eq::Fi} follows a common recursive procedure with energy, since it has the same structure as given in Eq.~\eqref{eq::33}, and the overall cost for evaluating force on all~$N$ particles for each~$k$ point also amounts to~$O(N)$, and the resulting SOEwald2D method is summarized in Algorithm~\ref{alg:SOEwald2D}.

Moreover, Lemma~\ref{lem::forceerr} establishes the overall error on forces $\bm{F}_{i}$, and the proof follows an almost similar approach to what was done for the energy. %The proof is postponed to~\ref{app::forceerror}

\begin{lem}\label{lem::forceerr}
	The total error of force by the SOEwald2D is given by
	\begin{equation}
		\mathscr{E}_{\bm{F}_{i}} := \mathscr{E}_{\bm{F}_{\emph{s}}^{i}} + \mathscr{E}_{\bm{F}_{\emph{l}}^{i}} + \sum_{\bm{h}\neq \bm{0}} \mathscr{E}_{\bm{F}_{\emph{l}}^i, \emph{SOE}}^{\bm{h}} + \mathscr{E}_{\bm{F}_{\emph{l}}^{i},\emph{SOE}}^{\bm{0}}
	\end{equation}
	where the first two terms are the truncation error and provided in Proposition~\ref{prop::2.12}. The remainder terms 
	\begin{equation}
		\mathscr{E}_{\bm{F}_{\emph{l}}^i,\emph{SOE}}^{\bm{h}} := \bm{F}_{\emph{l}}^{\bm{h}, i} - \bm{F}_{\emph{l},\emph{SOE}}^{\bm{h}, i}, \quad\emph{and} \quad \mathscr{E}_{\bm{F}_{\emph{l}}^{i}, \emph{SOE}}^{\bm{0}} := \bm{F}_{\emph{l}}^{\bm{0}, i}-\bm{F}_{\emph{l}, \emph{SOE}}^{\bm{0}, i}
	\end{equation}
	are the error due to the SOE approximation as Eqs.~\eqref{eq::Fi}-\eqref{eq:z-der}. Given SOE parameters $w_l$ and $s_l$ along with the ideal-gas assumption, one has the following estimate:
	\begin{equation}
		\sum_{\bm{h}\neq\bm{0}} \mathscr{E}_{\bm{F}_{\emph{l}}^i, \emph{SOE}}^{ \bm{h}}\leq \sqrt{2}\lambda_D^2\alpha^2q_{i}^2\varepsilon,\quad\text{and}\quad \mathscr{E}_{\bm{F}_{\emph{l}}^{i},\emph{SOE}}^{\bm{0}}\leq \frac{4\sqrt{\pi}\lambda_D^2 (1+2\alpha)L_z}{L_xL_y}q_{i}^2\varepsilon.
	\end{equation}
\end{lem}



%the electrostatic potential satisfies the Poisson's equation
%\begin{equation}
%-\Delta \phi(\bm{r})=4\pi \rho_{q}(\bm{r}),\quad\rho_{q}(\bm{{r}})=q\rho_+g_{++}(\bm{r})-q\rho_{-}g_{+-}(\bm{r}),
%\end{equation}
%where the charge density are expressed in terms of the charge-charge correlation functions $g_{++}(\bm{r})=g_{--}(\bm{r})(\bm{r})$ and $g_{+-}(\bm{r})=g_{-+}(\bm{r})$, and $\rho_+=\rho_-=\rho/2$ are average densities of positive and negative ions. In the work of Debye and H$\ddot{\text{u}}$ckel, the correlation functions are approximated by $g_{ij}(\bm{r})=e^{-\beta q_j\phi_i(\bm{r})}$, 

%\section{The SOE approximation error of force}\label{app::forceerror}

\section{Force expressions of RBE2D}\label{app::surfacecharge}

In this section, we will discuss the extension of the developed RBE algorithm to handle systems that involve both continuous surface charges and free ions. Without loss of generality, we assume two charged interfaces are located at $z=0$ and $z=H$ and with uniform surface charge density $\sigma_d$ and $\sigma_u$. The system satisfies the charge neutrality condition \begin{equation}\label{eq::chargeneutrality}
    \sum_{i=1}^N q_i+(\sigma_d+\sigma_u)L_xL_y=0\;,
\end{equation}
so that the electrostatic energy and force of the system are both well-defined. When dealing with surface charges, it is common in the literature to handle discrete ions and continuous surface charges separately~\cite{spohr1997effect,yi2017note,yuan2021particle}. The former is typically treated as an infinite summation problem, while the latter is often obtained by solving an additional Poisson equation. %However, during the energy calculation, some divergent terms may arise. Since these terms do not affect the forces, they are often disregarded in a rough manner. 

In this manuscript, we employ a unified methodology where the continuous charge is treated as the limit distribution of an infinite set of discrete charges.
To ensure accuracy, the parameters $M$ and $L_z$ are selected according to the parameter selecting strategy proposed in Section~\ref{sec:parameter}.
As a result, we can accurately describe the force exerted on each mobile ion as:
\begin{equation}
\bm{F}^{\text{c}}(\bm{r}_i)=\bm{F}^{\text{c}}_{\text{real}}(\bm{r}_i)+\bm{F}^{\text{c}}_{\text{Fourier}}(\bm{r}_i)+\bm{F}^{\text{c}}_{\text{wall}}(\bm{r}_i),
\end{equation}
including the real space, the Fourier space, and ion-wall components. The real space component is given by
\begin{equation}\label{eq::F_real_detail}
\begin{split}
\bm{F}_{\text{real}}^{\te{c}}(\bm{r}_i)
& := -q_i\sum_{j=1}^{N} q_j \Biggg[\sum_{\bm{n}}{}^\prime \frac{\left(\bm{r}_{i,\bm{n}} - \bm{r}_{j}\right)}{\abs{\bm{r}_{i,\bm{n}} - \bm{r}_{j}}} K(\abs{\bm{r}_{i,\bm{n}} - \bm{r}_{j}}) \\
&+ \frac{1}{2} \sum_{\bm{n}} \sum_{l=1}^{M} \frac{\gamma_{l}^+\left(\bm{r}_{i,\bm{n}}-\bm{r}_{j+}^{(l)}\right)}{\left|\bm{r}_{i,\bm{n}}-\bm{r}_{j+}^{(l)}\right|} K\left(\left|\bm{r}_{i,\bm{n}}-\bm{r}_{j+}^{(l)}\right|\right)\\
&+\frac{1}{2}\sum_{\bm{n}}\sum_{l=1}^{M}\frac{\gamma_{l}^-\left(\bm{r}_{i,\bm{n}}-\bm{r}_{j-}^{(l)}\right)}{\left|\bm{r}_{i,\bm{n}}-\bm{r}_{j-}^{(l)}\right|}K\left(\left|\bm{r}_{i,\bm{n}}-\bm{r}_{j-}^{(l)}\right|\right)\Biggg]\\
&-\frac{q_i}{2}\sum_{j=1}^{N}q_j\sum_{\bm{n}}\sum_{l=1}^{M}\begin{pmatrix}
1&0&0\\
0&1&0\\
0&0&(-1)^l
\end{pmatrix}\Biggg[
\frac{\gamma_{l}^+\left(\bm{r}_{i,\bm{n}}-\bm{r}_{j+}^{(l)}\right)}{\left|\bm{r}_{i,\bm{n}}-\bm{r}_{j+}^{(l)}\right|}K\left(\left|\bm{r}_{i,\bm{n}}-\bm{r}_{j+}^{(l)}\right|\right)\\
&+\frac{\gamma_{l}^-\left(\bm{r}_{i,\bm{n}}-\bm{r}_{j-}^{(l)}\right)}{\left|\bm{r}_{i,\bm{n}}-\bm{r}_{j-}^{(l)}\right|}K\left(\left|\bm{r}_{i,\bm{n}}-\bm{r}_{j-}^{(l)}\right|\right)\Biggg]
\end{split}
\end{equation}
where $\bm{r}_{i,\bm{n}}=\bm{r}_{i}+\bm{\ell}$, and $K(r):=\text{erfc}(\alpha r)/r^2+2\alpha e^{-\alpha^2r^2}/(\sqrt{\pi}r)$. The Fourier component of force, $\bm{F}^{\text{c}}_{\text{Fourier}}$, has been provided in Eq.~\eqref{eq::52}. In practice, we use the RBE force $\bm{F}^{\text{c},*}_{\text{Fourier}}$ in Eq.~\eqref{eq::important} as an unbiased estimator to $\bm{F}^{\text{c}}_{\text{Fourier}}$, in which the formula of IBC term is more complicated than  {in} the case of  {a} neutral interface:
\begin{equation}\label{eq::IBC}
\begin{split}
\bm{F}_{\text{IBC}}^{M}(\bm{r}_i)=&-\frac{2\pi q_{i}}{L_xL_yL_z}\left[\mathcal{M}_{z}\left(1+\sum_{l=1}^{M}(-1)^{l}\left(\gamma_{+}^{(l)}+\gamma_{-}^{(l)}\right)\right)+\mathcal{M}_{z}^{M}\right]\V{\hat e_z}\\
&+\frac{2\pi q_i}{L_xL_yL_z}\sum_{j=1}^{N}q_j\left[z_i+\sum_{l=1}^{M}(-1)^{l}\left(\gamma_{+}^{(l)}z_{i+}^{(l)}+\gamma_{-}^{(l)}z_{i-}^{(l)}\right)\right]\V{\hat e_z}\\
&+\frac{2\pi q_iz_i}{L_xL_yL_z}\sum_{j=1}^{N}q_j\left[1+\sum_{l=1}^{M}\left(\gamma_{+}^{(l)}+\gamma_{-}^{(l)}\right)\right]\V{\hat e_z},
\end{split}
\end{equation}
where the $z$-direction dipole moments are defined via
\begin{equation}
\mathcal{M}_{z}:=\sum_{i=1}^{N}q_iz_i\quad\text{and}\quad \mathcal{M}_{z}^{M}:=\sum_{i=1}^{N}q_i\left[z_{j}+\sum_{l=1}^{M}\left(\gamma_{+}^{(l)}z_{j+}^{(l)}+\gamma_{-}^{(l)}z_{j-}^{(l)}\right)\right].
\end{equation}
It is important to highlight that if the free ions in the solution continue to fulfill the charge neutrality condition Eq.~\eqref{eq::chargeneutrality}, the last two terms in Eq.~\eqref{eq::IBC} vanish. 
The ion-wall component represents the force exerts on the mobile ions due to the charged walls, and is given by
\begin{equation}
\bm{F}_{\text{wall}}^{\te{c}}:=-2\pi z_{i}q_{i}\left[\sigma_u-\sigma_d+\frac{(\sigma_u+\sigma_d)(\gamma_{\text{top}}-\gamma_{d})\left(1-\gamma_{\text{top}}^{\lceil M/2 \rceil}\gamma_{d}^{\lceil M/2 \rceil}\right)}{1-\gamma_{\text{top}}\gamma_{d}}\right]\V{\hat e_z}.
\end{equation}