\section{Fundamental results from Fourier analysis} \label{app::Fourier}
In this appendix, we state several fundamental results from Fourier analysis for doubly periodic functions, associated with the Fourier transform pair defined in Definition~\ref{Def::Fourier}. These results are useful for us, and their proofs are well established and can be referenced in classical literature, such as in the work of Stein and Shakarchi~\cite{stein2011fourier}.
\begin{lem}\label{lem::Convolution}
	(Convolution theorem) Let $f(\bm{\rho},z)$ and $g(\bm{\rho},z)$ be two functions which are periodic in $\bm{\rho}$ and non-periodic in $z$. Suppose that $f$ and $g$ have Fourier transform $\widetilde{f}$ and $\widetilde{g}$, respectively. Their convolution is defined by
	\begin{equation}\label{eq:Q2D_cov}
		u(\bm{\rho},z):=(f\ast g)(\bm{\rho},z)=\int_{\mathcal{R}^2}\int_{\mathbb{R}}f(\bm{\rho}-\bm{\rho}',z-z')g(\bm{\rho}',z')dz'd\bm{\rho}',
	\end{equation}
	satisfying
	\begin{equation}
		\widetilde{u}(\bm{k},\kappa)=\widetilde{f}(\bm{k},\kappa)\widetilde{g}(\bm{k},\kappa).
	\end{equation}
	
\end{lem}
\begin{lem}\label{lem::Poisson}
	(Poisson summation formula) Let $f(\bm{\rho},z)$ be a function which is periodic in $\bm{\rho}$ and non-periodic in $z$. Suppose that $f$ has Fourier transform $\widetilde{f}$ and $\bm{r}=(\bm{\rho},z)$. Then one has
	\begin{equation}
		\sum_{\bm{m}\in \mathbb{Z}^2} f(\bm{r} + \V{\mathcal{M}}) = \frac{1}{2\pi L_x L_y}\sum_{\bm{k}\in \mathcal{K}^2}\int_{\mathbb{R}}\widetilde{f}(\bm{k},\kappa)e^{\m{i} \bm{k}\cdot\bm{\rho}}e^{\m{i} \kappa z}d\kappa.
	\end{equation}
\end{lem}
\begin{lem}\label{lem::2dfourier}
	(Radially symmetric functions) 
	Suppose that $f(\rho,z)$ is periodic and radially symmetric in $\bm{\rho}$, i.e., $f(\bm{\rho},z)=f(\rho,z)$. Then its Fourier transform $\widetilde{f}$ is also radially symmetric. Indeed, one has
	\begin{equation}
		\widetilde{f}(\rho,z)=2\pi\int_{0}^{\infty}J_0(k\rho)f(\rho,z)\rho d\rho.
	\end{equation}
\end{lem}

% \gao{remove the Appendix A `Derivation of Ewald2D method'}

\section{Proof of Lemma~\ref{thm::SpectralExpansion}}\label{app::deriv}
By applying the Fourier transform to Poisson's equation
\begin{equation}\label{eq::B.1}
	-\Delta\phi_{\ell}(\bm{\rho},z)=4\pi g(\bm{\rho},z)\ast\tau(\bm{\rho},z),
\end{equation}
one obtains  
\begin{equation}\label{eq::B2}
	\widetilde{\phi}_{\ell}(\bm{k},\kappa)=\frac{4\pi}{k^2+\kappa^2}\widetilde{g}(\bm{k},\kappa)\widetilde{\tau}(\bm{k},\kappa)\quad\text{with}\quad \widetilde{g}(\bm{k},\kappa)=\sum_{j=1}^{N}q_{j} e^{-\m{i} \bm{k}\cdot\bm{\rho}_{j}}e^{-\m{i} \kappa z_{j}}
\end{equation}
via the convolution theorem and the Poisson summation formula (see Lemmas~\ref{lem::Convolution} and \ref{lem::Poisson}, respectively). Applying the inverse Fourier transform to Eq.~\eqref{eq::B2} yields
\begin{equation}\label{eq::phil}
	\phi_{\ell}(\bm{\rho},z)=\frac{2}{L_xL_y}\sum_{j=1}^{N}q_{j}\sum_{\bm{k}\neq\bm{0}}\int_{\mathbb{R}}\frac{e^{-(k^2+\kappa^2)/(4\alpha^2)}}{k^2+\kappa^2}e^{-\m{i} \bm{k}\cdot(\bm{\rho}-\bm{\rho}_{j})}e^{-\m{i} \kappa(z-z_{j})}d\kappa + \phi^{\bm{0}}_{\ell}(z),
\end{equation}
where $\phi^{\bm{0}}_{\ell}(z)$ is the contribution from zero mode. From \cite{oberhettinger2012tables}, one has 
\begin{equation}\label{eq::integral}
	\int_{\mathbb{R}} \frac{e^{-(k^2+\kappa^2)/(4\alpha^2)}}{k^2+\kappa^2} e^{-\m{i} \kappa z} d\kappa = \frac{\pi}{2k} \left[\xi^{+}(k,z)+\xi^{-}(k,z)\right]
\end{equation}
for $\bm{k}\neq\bm{0}$, where $\xi^{\pm}(k,z)$ are defined via Eq.~\eqref{eq::xi20}. Substituting Eq.~\eqref{eq::integral} into the first term of Eq.~\eqref{eq::phil} yields $\phi_{\ell}^{\bm{k}}(\bm{r})$ defined via Eq.~\eqref{eq:philk}.

By Theorem~\ref{order}, the zero-frequency term $\phi^{\bm{0}}_{\ell}(z)$ always exists and its derivation is very subtle. Let us apply the 2D Fourier transform (see Lemma~\ref{lem::2dfourier}) to Poisson's equation Eq.~\eqref{eq::B.1} only on periodic dimensions, and then obtain
\begin{equation}\label{eq::B.5}
	(-\partial_z^2+k^2)\widehat{\phi}_{\ell}(\bm{k},z)=4\pi\widehat{g}(\bm{k},z)\ast_{z}\widehat{\tau}(\bm{k},z),
\end{equation}
where $\ast_z$ indicates the convolution operator along $z$ dimension. Simple calculations suggest
\begin{equation}
	\widehat{g}(\bm{k},z)=\sum_{j=1}^{N}q_{j} e^{-\m{i} \bm{k}\cdot\bm{\rho}_{j}}\delta(z-z_{j}),\quad\text{and}\quad \widehat{\tau}(\bm{k},z)=\frac{\alpha}{\sqrt{\pi}} e^{-k^2/(4\alpha^2)}e^{-\alpha^2z^2}.
\end{equation}
The solution of Eq.~\eqref{eq::B.5} for $\bm{k}=\bm{0}$ can be written as the form of double integral that is only correct up to a linear mode,
\begin{equation}\begin{split}
		\phi_{\ell}^{\bm{0}}(z)&=-\frac{4\pi}{L_xL_y}\int_{-\infty}^{z}\int_{-\infty}^{z_1}\widehat{g}(\bm{0},z_2)\ast_{z_2}\widehat{\tau}(\bm{0},z_2)dz_2dz_1+A_0z+B_0\\
		&=-\frac{2\pi}{L_xL_y}\sum_{j=1}^{N}q_{j}\left[z-z_{j}+(z-z_{j})\erf\left(\alpha(z-z_{j})\right)+\frac{e^{-\alpha^2(z-z_{j})^2}}{\sqrt{\pi}\alpha}\right]+A_0z+B_0.
	\end{split}
\end{equation}
To analyze the short-range component $\phi_{s}(\bm{\rho},z)$ using a procedure similar to Eqs.~\eqref{eq::B.1}-\eqref{eq::integral}, one obtains
\begin{equation}\begin{split}
		\phi_{s}^{\bm{0}}(z) =& \frac{\pi}{L_x L_y} \sum_{j=1}^{N} \lim_{\bm{k}\rightarrow\bm{0}} \frac{1}{k} \left[2e^{-k|z|}-\xi^{+}(\bm{k},z) - \xi^{-}(\bm{k},z)\right]\\
		=& \frac{2\pi}{L_xL_y}\sum_{j=1}^{N}q_{j}\left[-|z-z_{j}|+(z-z_{j})\erf\left(\alpha(z-z_{j})\right)+\frac{e^{-\alpha^2(z-z_{j})^2}}{\sqrt{\pi}\alpha}\right].
	\end{split}
\end{equation}
Since $\phi_{s}^{\bm{0}}(z)+\phi_{\ell}^{\bm{0}}(z)$ matches the boundary condition Eq.~\eqref{eq::boundary1} as $z\rightarrow \pm\infty$ and by the charge neutrality condition, one solves 
\begin{equation}
	A_0 = \frac{2\pi}{L_xL_y}\sum_{j=1}^{N}q_{j}z \equiv 0,\quad \text{and}\quad B_0=-\frac{2\pi}{L_xL_y}\sum_{j=1}^{N}q_{j}z_{j}.
\end{equation}
This result finally gives 
\begin{equation}
	\phi_{\ell}^{\bm{0}}(z)=-\frac{2\pi}{L_xL_y}\sum_{j=1}^{N}q_{j}\left[(z-z_{j})\erf\left(\alpha(z-z_{j})\right)+\frac{e^{-\alpha^2(z-z_{j})^2}}{\sqrt{\pi}\alpha}\right].
\end{equation}

%and can be written as 
%\begin{equation}\phi^{\bm{0}}_{\ell}(\bm{\rho},z)=\frac{\pi}{L_x L_y}\sum_{j=1}^{N}q_{j}\lim_{\bm{k}\rightarrow \bm{0}}\frac{e^{\m{i} \bm{k}\cdot(\bm{\rho}-\bm{\rho}_{j})}}{k}\left[\xi^{+}(k,z)+\xi^{-}(k,z)\right].\end{equation} By Taylor expansion with respect to $\bm{k}$, one gets \begin{equation}\begin{split}\phi^{\bm{0}}_{\ell}(\bm{\rho},z)=&\frac{\pi}{L_x L_y}\sum_{j=1}^{N}q_{j}\lim_{\bm{k}\rightarrow \bm{0}}\frac{1+\m{i} \bm{k}\cdot(\bm{\rho}-\bm{\rho}_{j})+\mathcal{O}(k^2)}{k}\Bigg[2-2k(z-z_{j})\erf(\alpha (z-z_{j}))\\&-\frac{2ke^{-\alpha^2(z-z_{j})^2}}{\sqrt{\pi}\alpha}+\mathcal{O}(k^2)\Bigg]\\=&-\frac{2\pi}{L_xL_y}\sum_{j=1}^{N}q_{j}\left[(z-z_{j})\erf(\alpha (z-z_{j})-\frac{e^{-\alpha^2(z-z_{j})^2}}{\sqrt{\pi}\alpha}\right]\end{split}\end{equation}

% \section{Systems with charged slabs}\label{sec::sysslabs}
% In the presence of charged slabs, boundary layers naturally arise -- opposite ions accumulate near the interface, forming an electric double layer. The structure of electric double layers plays essential role for properties of interfaces and has caught much attention~\cite{messina2004effect,breitsprecher2014coarse,moreira2002simulations}. Since charges on the slabs are often represented as a continuous surface charge density, we present the Ewald2D formulation with such a situation can be well treated.

% Without loss of generality, one assumes that the two charged slab walls are located at $z=0$ and $z=L_z$ and with smooth surface charge densities $\sigma_{\mathrm{bot}}(\bm{\rho})$ and $\sigma_{\mathrm{top}}(\bm{\rho})$, respectively. Note that both $\sigma_{\mathrm{bot}}(\bm{\rho})$ and $\sigma_{\mathrm{top}}(\bm{\rho})$ are doubly-periodic according to the quasi-2D geometry. 
% Under such setups, the potential $\phi$ can be written as the sum of particle-particle and particle-slab contributions,
% \begin{align}
% 	\phi(\bm{r})=\phi_{\text{p-p}}(\bm{r})+\phi_{\text{p-s}}(\bm{r}).
% \end{align}
% Here, $\phi_{\text{p-p}}$ satisfies Eq.~\eqref{eq::Poisson} associated with the boundary condition Eq.~\eqref{eq::boundary2}. Note that Eq.~\eqref{eq::boundary1} does not apply since the particles are overall non-neutral. $\phi_{\text{p-s}}$ satisfies 
% \begin{equation}\label{eq::PoionWall}
% 	-\Delta \phi_{\text{p-s}}(\bm{r}) = 4\pi h(\bm{r}), \quad \text{with}~ h(\bm{r}) =  \sigma_{\mathrm{bot}}(\bm{\rho}) \delta(z) + \sigma_{\mathrm{top}}(\bm{\rho}) \delta(z-L_z),
% \end{equation}
% with the boundary condition
% \begin{equation}\label{eq::boundionwall}
% 	\lim_{z\rightarrow\pm\infty}\phi_{\text{p-s}}(\bm{r})=\mp \frac{2\pi}{L_xL_y}\left(\int_{\mathcal{R}^2}\sigma_{\mathrm{bot}}(\bm{\rho})|z|d\bm{\rho}+\int_{\mathcal{R}^2}\sigma_{\mathrm{top}}(\bm{\rho})|z-L_z|d\bm{\rho}\right)
% \end{equation}
% which is simply the continuous analog of Eq.~\eqref{eq::boundary2}. 

% % The charge neutrality condition of the system reads
% % \begin{equation}\label{eq::chargeneu}
% 	% \sum_{i=1}^{N}q_{i}+\int_{\mathcal{R}^2}\left[\sigma_{\mathrm{top}}(\bm{\rho})+\sigma_{\mathrm{bot}}(\bm{\rho})\right]d\bm{\rho}=0,
% 	% \end{equation}
% % which leads to the well-definedness of potential $\phi$ together with the DPBCs imposed. 
% The potential $\phi_{\text{p-p}}$ then follows immediately from Lemma~\ref{thm::SpectralExpansion}
% \begin{equation}\label{eq::phiion-ion}
% 	\phi_{\text{p-p}}(\bm{r}_{i})=\phi_{s}(\bm{r}_{i}) + \sum_{\bm{k}\neq\bm{0}} \phi_{\ell}^{\V{k}}(\bm{r}_{i}) + \phi_{\ell}^{\V{0}}(\bm{r}_{i}) - \phi_{\text{self}}^{i},
% \end{equation}
% with each components given by Eqs.~\eqref{eq:phi_s}, \eqref{eq:philk}, \eqref{eq:phil0}, and \eqref{eq::self}, respectively. 
% The 2D Fourier series expansion of~$\phi_{\text{p-s}}$ is provided in the following Theorem~\ref{thm::ionwall}, where its convergence rate is controlled by the smoothness of surface charge densities.
% \begin{thm}\label{thm::ionwall}
% 	Suppose that $\widehat{\sigma}_{\mathrm{bot}}$ and $\widehat{\sigma}_{\mathrm{top}}$ are two-dimensional Fourier transform (see Lemma~\ref{lem::2dfourier}) of $\sigma_{\mathrm{bot}}$ and $\sigma_{\mathrm{top}}$, respectively. By Fourier analysis, the particle-slab component of the electric potential is given by
% 	\begin{equation}\label{eq::phiionwall}
% 		\phi_{\emph{p-s}}(\bm{r}_{i}) = \frac{2\pi}{L_xL_y}\sum_{\bm{k}\neq \bm{0}}\frac{e^{\m{i} \bm{k}\cdot\bm{\rho}_{i}}}{k}\left[\widehat{\sigma}_{\mathrm{bot}}(\bm{k})e^{-k|z_{i}|}+\widehat{\sigma}_{\mathrm{top}}(\bm{k})e^{-k|z_{i}-L_z|}\right]+\phi_{\emph{p-s}}^{\bm{0}}(\bm{r}_{i})\;,
% 	\end{equation}
% 	where the 0-th mode reads
% 	\begin{equation}\label{eq::phionwallzero}
% 		\phi_{\emph{p-s}}^{\bm{0}}(\bm{r}_{i})=-\frac{2\pi}{L_xL_y}\Big[\widehat{\sigma}_{\mathrm{bot}}(\bm{0})|z_{i}|+\widehat{\sigma}_{\mathrm{top}}(\bm{0})|z_{i}-L_z|\Big]\;.
% 	\end{equation}
% \end{thm}
% \begin{proof}
% 	For $\bm{k}\neq\bm{0}$, applying the quasi-2D Fourier transform to both sides of Eq.~\eqref{eq::PoionWall} yields
% 	\begin{equation}
% 		\widetilde{\phi}_{\text{p-s}}(\bm{k},\kappa)=\frac{4\pi}{k^2+\kappa^2}\left[\widehat{\sigma}_{\mathrm{bot}}(\bm{k})+\widehat{\sigma}_{\mathrm{top}}(\bm{k})e^{-\m{i} \kappa L_z}\right]\;.
% 	\end{equation}
% 	For $\bm{k}=0$, one first applys the 2D Fourier transform in $xy$ to obtain
% 	\begin{equation}
% 		\left(-\partial_z^2+k^2\right)\widehat{\phi}(\bm{k},z)=4\pi\left[\widehat{\sigma}_{\mathrm{bot}}(\bm{k})\delta(z)+\widehat{\sigma}_{\mathrm{top}}(\bm{k})\delta(z-L_z)\right]\;.
% 	\end{equation}
% 	By integrating both sides twice and taking $\bm{k}=0$, the $0$-th mode follows 
% 	\begin{equation}
% 		\widehat{\phi}(\bm{0},z)=-2\pi\left[\widehat{\sigma}_{\mathrm{bot}}(\bm{0})|z|+\widehat{\sigma}_{\mathrm{top}}(\bm{0})|z-L_z|\right]+A_0z+B_0\;,
% 	\end{equation}
% 	where $A_0$ and $B_0$ are undetermined constants. 
% 	Finally, applying the corresponding inverse transforms to $\widetilde{\phi}_{\text{p-s}}(\bm{k},\kappa)$ and $\widehat{\phi}(\bm{0},z)$ such that the boundary conditions Eq.~\eqref{eq::boundionwall} is matched, one has $A_0=B_0=0$. The proof of Eqs.~\eqref{eq::phiionwall}-\eqref{eq::phionwallzero} is then completed. 
% \end{proof}


% Consider the ideal case that both $\sigma_{\mathrm{bot}}$ and $\sigma_{\mathrm{top}}$ are uniformly distributed. This simple setup is widely used in many studies on interface properties. Since in this case all nonzero modes vanish, one has
% \begin{equation}\label{eq:spectial}
% 	\phi_{\text{p-s}}(\bm{r}_{i})=\phi_{\text{p-s}}^{\bm{0}}(\bm{r}_{i})=-2\pi\left[\sigma_{\mathrm{top}}(L_z - z_{i})+\sigma_{\mathrm{bot}}(z_{i} - 0))\right]\;,
% \end{equation}
% for all $z_{i}\in [0, L_z]$. 
% Here zero is retained to indicate the location of bottom slab. 

% For completeness, Proposition \ref{welldefinedness} provides the result of the well-definedness.
% \begin{prop} \label{welldefinedness}
% 	The total electrostatic potential $\phi$ is well-defined. 
% \end{prop}
% \begin{proof}
% 	For any finite $z$, $\phi$ is clearly well defined. Consider the case of $z\rightarrow \pm \infty$.
% 	By boundary conditions ~\eqref{eq::boundary2} and \eqref{eq::boundionwall} and the charge neutrality condition Eq.~\eqref{eq::chargeneu}, one has
% 	\begin{equation}
% 		\begin{split}
% 			\lim_{z\rightarrow \pm \infty}\phi(\bm{r})&=\lim_{z\rightarrow \pm \infty}\left[\phi_{\text{p-p}}(\bm{r})+\phi_{\text{p-s}}(\bm{r})\right]\\
% 			&=\pm\frac{2\pi}{L_xL_y}\left[\sum_{j=1}^{N}q_{j}z_{j}+\int_{\mathcal{R}^2}\left(0\sigma_{\mathrm{bot}}(\bm{\rho})+L_z\sigma_{\mathrm{top}}(\bm{\rho})\right)d\bm{\rho}\right]
% 		\end{split}
% 	\end{equation}
% 	which is a finite constant. Thus the proof is completed.
% \end{proof}

% For the the particle-slab interaction formulation, we observe a constant discrepancy between Eq.~\eqref{eq:spectial} derived here and those in literature~\cite{dos2017simulations,10.1063/1.4998320}. 
% It is because here one starts with the precise Ewald2D summation approach, different from the approach of employing approximation techniques to transform the original doubly-periodic problem into a triply-periodic problem first, and subsequently introducing charged surfaces. 
% This constant discrepancy makes no difference in force calculations for canonical ensembles. 
% However, for simulations under isothermal-isobaric ensembles, this $L_z$-dependent value is important for the pressure calculations~\cite{li2024noteaccuratepressurecalculations}. 
% And one should use Eq.~\eqref{eq:spectial} derived here for correct simulations.

% Based on the expression of electrostatic potential $\phi$ derived above, the total electrostatic energy can be computed via the Ewald2D summation formula:  
% \begin{align}\label{eq::34}
% 	U = U_{\text{p-p}} + U_{\text{p-s}}, \quad \text{with} \quad U_{\text{p-p}} := U_{s} + \sum_{\bm{k}\neq\bm{0}}U_{\ell}^{\bm{k}}+U_{\ell}^{\bm{0}}- U_{\text{self}}\;,
% \end{align}
% where $U_*=\sum_{i}\phi_*$ with $*$ representing any of the subscripts used in Eq.~\eqref{eq::34}.

 \section{The ideal-gas assumption for error analysis}\label{app::ideal-gas}
 Let $\bm{\mathcal{\psi}}$ represent a statistical quantity in an interacting particle system, and we aim to analyze its root mean square value given by
 \begin{equation}\label{eq::deltaS}
 	\delta \bm{\mathcal{\psi}}:=\sqrt{\frac{1}{N}\sum_{i=1}^{N}\|\bm{\mathcal{\psi}}_{i}\|^2},
 \end{equation}
 where $\bm{\mathcal{S}}_{i}$ denotes the quantity associated with particle $i$ (e.g., energy for one dimension or force for three dimensions). Assume that $\bm{\mathcal{\psi}}_{i}$ takes the form
\begin{equation}
	\bm{\mathcal{\psi}}_{i}=q_{i} \sum_{j \neq i} q_{j} \bm{\zeta}_{i j},
\end{equation}
due to the superposition principle of particle interactions, which implies that the total effect on particle $i$ can be expressed as the sum of contributions from each $i-j$ pair (including periodic images). Here, $\bm{\zeta}_{i j}$ represents the interaction between two particles. The ideal-gas assumption leads to the following relation
\begin{equation}
	\left\langle\boldsymbol{\zeta}_{i j} \boldsymbol{\zeta}_{i k}\right\rangle=\delta_{j k}\left\langle\boldsymbol{\zeta}_{i j}^2\right\rangle:=\delta_{j k} \zeta^2,
\end{equation}
where the expectation is taken over all particle configurations, and $\zeta$ is a constant. This assumption indicates that any two different particle pairs are uncorrelated, and the variance of each pair is expected to be uniform. In the context of computing the force variance of a charged system, this assumption implies that
\begin{equation}
	\left\langle\|\bm{\mathcal{\psi}}_{i}\|^2\right\rangle=q_{i}^2 \sum_{j, k \neq i} q_{j} q_k\left\langle\boldsymbol{\zeta}_{i j} \boldsymbol{\zeta}_{i k}\right\rangle \approx q_{i}^2 \zeta^2 Q,
\end{equation}
where $Q$ represents the total charge of the system. By applying the law of large numbers, one obtains $\delta\mathcal{\psi}\approx \zeta Q/\sqrt{N}$, which can be utilized for the mean-field estimation of the truncation error.

\section{Proof of Theorem~\ref{thm:ewald2d_phi_error}}\label{app:phierr}
We begin by considering the real space truncation error of electrostatic potential 
\begin{equation}
	\mathscr{E}_{\phi_{s}}(r_c,\alpha)(\bm{r}_{i}) = \sum_{|\bm{r}_{ij} + \V{\mathcal{M}}|>r_c}q_{j} \frac{\erfc(\alpha|\bm{r}_{ij} + \V{\mathcal{M}}|)}{|\bm{r}_{ij} + \V{\mathcal{M}}|}
\end{equation}
for $i$th particle, which involves neglecting interactions beyond $r_c$. By the analysis in \ref{app::ideal-gas}, this part of error can be approximated by $\delta\mathscr{E}_{\phi_{s}}$ with 
\begin{equation}\label{eq::delta^2phi}
	\delta^2\mathscr{E}_{\phi_{s}}=  \frac{1}{V}\sum_{j=1}^{N}q_{j}^2\int_{r_c}^{\infty}\frac{\erfc(\alpha r)^2}{r^2}4\pi r^2dr=\frac{4\pi Q}{V}\mathscr{Q}_{\emph{s}}(\alpha,r_c),
\end{equation}
where $\mathscr{Q}_{\emph{s}}(\alpha,r_c)$ is defined via Eq.~\eqref{eq::Qs2}. Note that the $\erfc(r)$ function satisfies (\cite{olver1997asymptotics}, pp. 109-112)
\begin{equation}\label{eq::asyerfc}
	\erfc(r)=\frac{e^{-r^2}}{\sqrt{\pi}}\sum_{m=0}^{\infty}(-1)^{m}\left(\frac{1}{2}\right)_mz^{-(2m+1)}
\end{equation}
as $r\rightarrow \infty$, where $(x)_m=x(x-1)\cdots(x-m+1)=x!/(x-m)!$ denotes the Pochhammer's symbol. Substituting Eq.~\eqref{eq::asyerfc} into Eq.~\eqref{eq::delta^2phi} and truncating at $m=1$ yields Eq.~\eqref{eq::Qs}. 

The Fourier space error, by~\ref{app::deriv}, is given by
\begin{equation}
	\mathscr{E}_{\phi_{\ell}}(k_c,\alpha)(\bm{r}_{i})=\frac{2}{L_xL_y}\sum_{j=1}^{N}q_{j}\sum_{|\bm{k}|>k_c}\int_{\mathbb{R}}\frac{e^{-(k^2+\kappa^2)/(4\alpha^2)}}{k^2+\kappa^2}e^{-\m{i} \bm{k}\cdot(\bm{\rho}-\bm{\rho}_{j})}e^{-\m{i} \kappa(z-z_{j})}d\kappa.
\end{equation}
For a large $k_c$, one can safely replace the truncation condition with $|\bm{k}+\kappa|>k_c$, resulting in
\begin{equation}
	\begin{split}
		\mathscr{E}_{\phi_{\ell}}(k_c,\alpha)(\bm{r}_{i})&\approx \frac{1}{2\pi^2}\sum_{j=1}^{N}q_{j} \int_{k_c}^{\infty}\int_{-1}^{1}\int_{0}^{2\pi}e^{-k^2/(4\alpha^2)}e^{-i kr_{ij}\cos\varphi}d\theta d\cos\varphi dk\\
		&=\frac{2}{\pi}\int_{k_c}^{\infty}\sum_{j=1}^{N}q_{j}\frac{\sin(kr_{ij})}{kr_{ij}}e^{-k^2/(4\alpha^2)}dk.
	\end{split}
\end{equation}
Here, the summation over Fourier modes is approximated using an integral similar to Eq.~\eqref{eq::integral2}, and one chooses a specific $(k,\theta,\varphi)$ so that the coordinate along $\cos\theta$ of $\bm{k}$ is in the direction of a specific vector $\bm{r}$, and $\bm{k}\cdot\bm{r}=kr \cos\varphi$.
The resulting formula is identical to Eq.~(21) in \cite{kolafa1992cutoff} for the fully-periodic case, and $\delta \mathscr{E}_{\phi_{\ell}}$ can be derived following the approach in \cite{kolafa1992cutoff}. 