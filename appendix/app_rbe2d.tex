

% \section{High-performance implementation}\label{sec::ImpleStr}
% The RBE2D method is highly suitable for parallelization and vectorization. In this section, we present our implementation strategy for the RBE2D, incorporating MPI parallelization and Intel 512-bit SIMD (AVX-512 architecture) for vectorization. The latter enables simultaneous calculations of sixteen neighbors for single-precision floating-point operations (or eight for double precision). Intel One-API is employed for parallelization, encompassing MPI and AVX-512 instructions. The approach presented here is equally applicable to other parallelization models and vectorization instructions.

% Our implementation contributes in two main aspects. Firstly, we design communication operations that overlap with computations, reducing the overall serial portion. Secondly, we reformulate the expressions for the Fourier component and the dielectric correction component of the Coulomb interaction to reduce computational complexity. We discuss the techniques and insights as follows. Although these details may seem unrelated to the scientific aspects, we believe they are crucial for readers who are interested to reproduce or even improve upon our work.

% Due to the mini-batch strategy in the Fourier space, a serial importance sampling procedure and a global broadcast operation are required at each MD step. However, we mitigate this cost through the designed non-jammed communication, computation/communication overlapping, and parallel execution. Here are the details: assuming $\mathcal{M}$ MPI ranks are employed, $\mathcal{M}$ independent sampling processes are executed in parallel within each rank. The $1$st MPI rank broadcasts the samples to other ranks using a blocking operation. Concurrently, the computation step of the Coulomb interaction is executed while the samples in the $2$nd MPI rank are being broadcasted. This sequence is repeated $\mathcal{M}-1$ times, followed by a new sampling loop. This strategy evaluates and updates the samples every $\mathcal{M}$ steps, significantly reducing the sampling and global communication costs. It is crucial to emphasize that in order to generate independent samples, it is necessary for each rank to utilize a distinct initial random number seed.

% The image charge reflection, which accounts for dielectric jumps, is efficiently handled by precomputing the coefficients, as described in Section~\ref{sec::imagecharge}. This step, which incurs substantial computational resources in previous FFT-based methods~\cite{yuan2021particle}, involves lower additional costs in the RBE2D method developed in this paper.

% Furthermore, the speedup of the RBE2D is limited by the cost of the real-space calculation, specifically the computation of $F_i^{\text{real}}$. To enhance efficiency in the real space, the RBE2D implementation adopts the procedure from LAMMPS \cite{plimpton1995fast}, including domain decomposition, cell lists, communication design, and data structures. The directive-based offload technique \cite{brown2015optimizing} is employed for further acceleration. The computationally expensive error complementary function is efficiently computed using a combination of Taylor expansion, bit masking, and table look-up techniques, following the approach of LAMMPS.

% In many works that employ FFT-based methods as their electrostatic solvers, the real-space cutoff is balanced to ensure that the costs of the real and Fourier spaces are approximately equal. For the RBE2D, the real-space cutoff should be made smaller when accelerating calculations in the Fourier space. By combining other optimized techniques for real-space cutoffs with the RBE2D, even better acceleration can be achieved. Our ongoing project involves the combination of RBE2D with the random batch list algorithm \cite{liang2021random}, which is a recently developed acceleration technique for the real-space part.

