\section{Force expression of the SOEwald2D} \label{app::force}

The Fourier component of force acting on the $i$th particle can be evaluated by taking the gradient of the energy with respect to the particle's position vector $\bm{r}_{i}$,
\begin{equation}\label{eq::Fi}
	\V{F}_{\ell}^i  \approx \V{F}^{i}_{\text{l},\text{SOE}} = -\grad_{\V{r}_{i}} U_{\ell,\text{SOE}} =  -\sum_{\bm{k}\neq \bm{0}} \grad_{\V{r}_{i}}U_{\ell,\text{SOE}}^{\V{k}} -\grad_{\V{r}_{i}} U_{\ell,\text{SOE}}^{\V{0}}
\end{equation}
where
\begin{align}    
	\grad_{\V{r}_{i}}U_{\ell,\text{SOE}}^{\V{k}} &= - \frac{\pi q_{i}}{L_x L_y} \left[ \sum_{1 \leq j < i} q_{j} \grad_{\V{r}_{i}} \varphi_{\text{SOE}}^{\V{k}}(\V{r}_{i}, \V{r}_{j}) +  \sum_{i < j \leq N} q_{j} \grad_{\V{r}_{i}} \varphi_{\text{SOE}}^{\V{k}}(\V{r}_{j}, \V{r}_{i}) \right]\;,\\
	\grad_{\V{r}_{i}} U_{\ell,\text{SOE}}^{\V{0}} &= - \frac{2 \pi q_{i}}{L_x L_y} \left[ \sum_{1 \leq j < i} q_{j} \grad_{\V{r}_{i}} \varphi^{\bm{0}}_{\text{SOE}}(\bm{r}_{i}, \bm{r}_{j}) + \sum_{i < j \leq N} q_{j} \grad_{\V{r}_{i}} \varphi^{\bm{0}}_{\text{SOE}}(\bm{r}_{j}, \bm{r}_{i})\right]\;.
\end{align}
Using the approximation Eqs.~\eqref{eq::SOEphi},~\eqref{eq::dz_plus} and~\eqref{eq::dz_minus}, one can write the derivative in periodic directions as
\begin{equation}
	\begin{split}
		\partial_{\bm{\rho}_{i}} \varphi_{\text{SOE}}^{\V{k}}(\V{r}_{i}, \V{r}_{j}) & = \frac{\m{i} \V{k} e^{ \m{i} \V{k} \cdot \V{\rho}_{ij}}}{k} \left[\xi^{+}_M(k, z_{ij})+\xi^{-}_M(k, z_{ij})\right]\\
		&=\frac{2\alpha e^{-k^2/(4\alpha^2)}}{\sqrt{\pi}k} \m{i} \bm{k}e^{\m{i} \V{k} \cdot \V{\rho}_{ij}} \sum_{\ell = 1}^{M}  \frac{w_l}{\alpha^2 s_l^2 - k^2}\left( 2 \alpha s_l e^{-k z_{ij}} - 2 k e^{-\alpha s_l z_{ij}}\right),
	\end{split}
\end{equation}
and in~$z$ direction as
\begin{equation}\label{eq:z-der}
	\begin{split}
		\partial_{z_{i}} \varphi_{\text{SOE}}^{\V{k}}(\V{r}_{i}, \V{r}_{j}) & = \frac{e^{\m{i} \V{k} \cdot \V{\rho}_{ij}}}{k} \left[\partial_{z_{i}} \xi^{+}_M(k, z_{ij}) + \partial_{z_{i}} \xi^{-}_M(k, z_{ij})\right]\\
		&=\frac{2 \alpha e^{-k^2/(4\alpha^2)}}{\sqrt{\pi}}e^{ \m{i} \V{k} \cdot \V{\rho}_{ij}} \sum_{\ell = 1}^{M}  \frac{w_l}{\alpha^2 s_l^2 - k^2}\left( - 2 \alpha s_l e^{-k z_{ij}} + 2 \alpha s_l e^{- \alpha s_l z_{ij}}\right)\;.
	\end{split}
\end{equation}
The partial derivatives of zero-frequency mode with respect to the periodic directions are zero, and the SOE approximation of its $z$-derivative is given by
\begin{equation}\label{eq:dzphi_0}
	\begin{split}
		\partial_{z_{i}} \varphi^{\bm{0}}_{\text{SOE}}(\bm{r}_{i},\bm{r}_{j}) 
		& = \sum_{l=1}^{M} \frac{w_l}{\sqrt{\pi}} \partial_{z_{i}} \left[\frac{2z_{ij}}{s_l}+\left(\frac{1}{\alpha} - \frac{2z_{ij}}{s_l}\right)e^{-\alpha s_l z_{ij}}\right] \\
		& = \sum_{l=1}^{M} \frac{w_l}{\sqrt{\pi}} \left[ \frac{2}{s_l} - \left( s_l + \frac{2}{s_l} - 2 \alpha z_{ij} \right) e^{-\alpha s_l z_{ij}} \right]\;.
	\end{split}
\end{equation}

It is important to note that the computation of Fourier space forces using Eq.~\eqref{eq::Fi} follows a common recursive procedure with energy, since it has the same structure as given in Eq.~\eqref{eq::33}, and the overall cost for evaluating force on all~$N$ particles for each~$k$ point also amounts to~$\mathcal{O}(N)$, and the resulting SOEwald2D method is summarized in Algorithm~\ref{alg:SOEwald2D}.

Moreover, Lemma~\ref{lem::forceerr} establishes the overall error on forces $\bm{F}_{i}$, and the proof follows an almost similar approach to what was done for the energy. %The proof is postponed to~\ref{app::forceerror}

\begin{lem}\label{lem::forceerr}
	The total error of force by the SOEwald2D is given by
	\begin{equation}
		\mathscr{E}_{\bm{F}_{i}} := \mathscr{E}_{\bm{F}_{\emph{s}}^{i}} + \mathscr{E}_{\bm{F}_{\emph{l}}^{i}} + \sum_{\bm{k}\neq \bm{0}} \mathscr{E}_{\bm{F}_{\emph{l}}^i, \emph{SOE}}^{\bm{k}} + \mathscr{E}_{\bm{F}_{\emph{l}}^{i},\emph{SOE}}^{\bm{0}}
	\end{equation}
	where the first two terms are the truncation error and provided in Proposition~\ref{prop::2.12}. The remainder terms 
	\begin{equation}
		\mathscr{E}_{\bm{F}_{\emph{l}}^i,\emph{SOE}}^{\bm{k}} := \bm{F}_{\emph{l}}^{\bm{k}, i} - \bm{F}_{\emph{l},\emph{SOE}}^{\bm{k}, i}, \quad\emph{and} \quad \mathscr{E}_{\bm{F}_{\emph{l}}^{i}, \emph{SOE}}^{\bm{0}} := \bm{F}_{\emph{l}}^{\bm{0}, i}-\bm{F}_{\emph{l}, \emph{SOE}}^{\bm{0}, i}
	\end{equation}
	are the error due to the SOE approximation as Eqs.~\eqref{eq::Fi}-\eqref{eq:z-der}. Given SOE parameters $w_l$ and $s_l$ along with the ideal-gas assumption, one has the following estimate:
	\begin{equation}
		\sum_{\bm{k}\neq\bm{0}} \mathscr{E}_{\bm{F}_{\emph{l}}^i, \emph{SOE}}^{ \bm{k}}\leq \sqrt{2}\lambda_D^2\alpha^2q_{i}^2\varepsilon,\quad\text{and}\quad \mathscr{E}_{\bm{F}_{\emph{l}}^{i},\emph{SOE}}^{\bm{0}}\leq \frac{4\sqrt{\pi}\lambda_D^2 (1+2\alpha)L_z}{L_xL_y}q_{i}^2\varepsilon.
	\end{equation}
\end{lem}

\section{The Debye-H$\ddot{\text{u}}$ckel approximation}\label{app::Debye}
Under the DH approximation, one is able to estimate functions associated with the $i$-th particle in the form:
\begin{equation}
	\mathscr{G}(\bm{r}_i)=\sum_{j\neq i}q_{j}e^{\m{i} \bm{k}\cdot\bm{\rho}_{ij}}f(z_{ij}),
\end{equation}
where $|f(z_{ij})|$ is bounded by a constant $C_f$ independent of $z_{ij}$. The DH theory considers the simplest model of an electrolyte solution confined to the simulation cell, where all $N$ ions are idealized as hard spheres of diameter $r_{a}$ carrying charge $\pm q$ at their centers. The charge neutrality condition requires that $N_+=N_-=N/2$. Let us fix one ion of charge $+q$ at the origin $r=0$ and consider the distribution of the other ions around it.

In the region $0<r\leq r_{a}$, the electrostatic potential $\phi(\bm{r})$ satisfies the Laplace equation $-\Delta\phi(\bm{r})=0$. For $r\geq r_{a}$, the charge density of each species is described by the Boltzmann distribution $\rho_{\pm}(\bm{r})=\pm qe^{\mp\beta q\phi(\bm{r})}\rho_r/2$ with number density $\rho_r=N/V$. In this region, the electrostatic potential satisfies the linearized Poisson-Boltzmann equation~\cite{levin2002electrostatic}:
\begin{equation}
	-\Delta \phi(\bm{r})=2\pi\left[q \rho_r e^{-\beta q\phi(\bm{r})}-q\rho_r e^{+\beta q\phi(\bm{r})}\right]\approx -4\pi \beta q^2\rho_r\phi(\bm{r}),
\end{equation}
and its solution is given by
\begin{equation}
	\phi(\bm{r})=\begin{cases}
		\dfrac{q}{4\pi r}-\dfrac{q\kappa}{4\pi (1+\kappa a)},& r<r_{a},\\[1em]
		\dfrac{qe^{\kappa a}e^{-\kappa r}}{4\pi r(1+\kappa a)},&r\geq r_{a},
	\end{cases}
\end{equation}
where $\kappa=\sqrt{\beta q^2\rho}$ denotes the inverse of Debye length $\lambda_{\text{D}}$. By this definition, the net charge density for $r>r_{a}$ is $\rho_>(\bm{r})=-\kappa^2\phi(\bm{r})$. Let us fix $\bm{r}_i$ at the origin. Given these considerations, for $r\geq r_a$, one obtains the following estimate:
\begin{equation}\label{eq::E.4}
	\begin{split}
		|\mathscr{G}(\bm{r}_i)|&\approx \left|\int_{\mathbb{R}^3\backslash B(\bm{r}_{i}, r_a)}\rho_>(\bm{r})e^{-\m{i} \bm{k}\cdot\bm{\rho}}f(z)d\bm{r}\right|\\
		&\leq \frac{q_iC_fe^{\kappa a}}{4\pi(1+\kappa a)}\int_{a}^{\infty}\frac{e^{-\kappa r}}{r}4\pi r^2dr\\
		&=q_iC_f\lambda_{\text{D}}^2.
	\end{split}
\end{equation}

It is remarked that upper bound Eq.~\eqref{eq::E.4} is derived under the continuum approximation. In the presence of surface charges, the charge distribution along the $z$-direction may lack spatial uniformity. However, due to the confinement of particle distribution between two parallel plates, the integral in Eq.~\eqref{eq::E.4} along the $z$-direction remains bounded. An upper bound in the form of $|\mathscr{G}(\bm{r}_i)|\leq C_s C_{f}q_i$ can still be expected, where $C_s$ is a constant related to the thermodynamic properties of the system.

%the electrostatic potential satisfies the Poisson's equation
%\begin{equation}
%-\Delta \phi(\bm{r})=4\pi \rho_{q}(\bm{r}),\quad\rho_{q}(\bm{{r}})=q\rho_+g_{++}(\bm{r})-q\rho_{-}g_{+-}(\bm{r}),
%\end{equation}
%where the charge density are expressed in terms of the charge-charge correlation functions $g_{++}(\bm{r})=g_{--}(\bm{r})(\bm{r})$ and $g_{+-}(\bm{r})=g_{-+}(\bm{r})$, and $\rho_+=\rho_-=\rho/2$ are average densities of positive and negative ions. In the work of Debye and H$\ddot{\text{u}}$ckel, the correlation functions are approximated by $g_{ij}(\bm{r})=e^{-\beta q_j\phi_i(\bm{r})}$, 

%\section{The SOE approximation error of force}\label{app::forceerror}

\section{The Metropolis algorithm} \label{app::Metropolis}

In practice, the Metropolis algorithm~\cite{metropolis1953equation, hastings1970monte} is employed to generate a sequence $\{\bm{k}_{\eta}\}_{\eta=1}^{P}$ from $h(\bm{k})$. 
Since $\bm{k}\circ \bm{L}=2\pi \bm{m}$ with $\bm{m}$ an integer vector, one can conveniently sample from the discrete distribution $\mathcal{H}(\bm{m})=h(\bm{k})$ to equivalently generate $\bm{k}$. 
Once the current state of the Markov chain $\bm{m}_{\eta}=\bm{m}^{\text{old}}$ is known, the algorithm generates a random variable $\bm{m}^*$ with $m_{\xi}^*\sim \mathcal{N}[0,(\alpha L_{\xi})^2/2\pi^2]$, which is the normal distribution with mean zero and variance $(\alpha L_{\xi})^2/2\pi^2$. The new proposal is taken as $\bm{m}^{\text{new}}=\text{round}(m_{x}^*,m_{y}^*)$. To determine the acceptance rate, one obtains the proposal probability 
\begin{equation}
	q(\bm{m}^{\text{new}}|\bm{m}^{\text{old}})=\prod_{\xi\in\{x,y\}}q(m^{\text{new}}_{\xi}|m^{\text{old}}_{\xi})
\end{equation}
where
\begin{equation}
	\begin{split}
		q(m^{\text{new}}_{\xi}|m^{\text{old}}_{\xi})&=\sqrt{\frac{\pi}{(\alpha L_{\xi})^2}}\int_{m^{\text{new}}_{\xi}-\frac{1}{2}}^{m^{\text{new}}_{\xi}+\frac{1}{2}}e^{-\pi^2t^2/(\alpha L_{\xi})^2}dt\\[1em]
		&=\begin{cases}
			\erf\left(\dfrac{\pi}{2\alpha L_{\xi}}\right),&m_{\xi}^{\text{new}}=0,\\[1.15em]
			\dfrac{1}{2}\left[\erf\left(\dfrac{\pi(2|m_{\xi}^{\text{new}}|+1)}{2\alpha L_{\xi}}\right)-\erf\left(\dfrac{\pi(2|m_{\xi}^{\text{new}}|-1)}{2\alpha L_{\xi}}\right)\right],&m_{\xi}^{\text{new}}\neq 0.
		\end{cases}
	\end{split}
\end{equation}
It is worth noting that the proposal distribution $q(\bm{m}^{\text{new}}|\bm{m}^{\text{old}})$ in the Metropolis algorithm presented here does not depend on the current state $\bm{m}^{\text{old}}$. The Metropolis acceptance probability is computed using the formula:
\begin{equation}
	a(\bm{m}^{\text{new}}|\bm{m}^{\text{old}}):=\min\left\{\frac{\mathcal{H}(\bm{m}^{\text{new}})q(\bm{m}^{\text{old}}|\bm{m}^{\text{new}})}{\mathcal{H}(\bm{m}^{\text{old}})q(\bm{m}^{\text{new}}|\bm{m}^{\text{old}})},1\right\}.
\end{equation}
If the proposal is rejected, then $\bm{m}_{\eta+1}=\bm{m}_{\eta}$. If $\bm{m}^{\text{new}}$ is accepted, then $\bm{m}_{\eta+1}=\bm{m}^{\text{new}}$. The sampling procedure has a small error since $\mathcal{H}(\bm{m}^{\text{new}})\approx q(\bm{m}^{\text{new}}|\bm{m}^{\text{old}})$. Our numerical experiments show an average acceptance rate of over $90\%$. Additionally, one can set an integer downsampling rate $\mathscr{D}$, where only one sample is taken from every $\mathscr{D}$ samples, to reduce the correlation between batches in the Metropolis process.