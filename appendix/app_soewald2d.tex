\section{Force expression of the SOEwald2D} \label{app::force}

The Fourier component of force acting on the $i$th particle can be evaluated by taking the gradient of the energy with respect to the particle's position vector $\bm{r}_{i}$,
\begin{equation}\label{eq::Fi}
	\V{F}_{\ell}^i  \approx \V{F}^{i}_{\text{l},\text{SOE}} = -\grad_{\V{r}_{i}} U_{\ell,\text{SOE}} =  -\sum_{\bm{k}\neq \bm{0}} \grad_{\V{r}_{i}}U_{\ell,\text{SOE}}^{\V{k}} -\grad_{\V{r}_{i}} U_{\ell,\text{SOE}}^{\V{0}}
\end{equation}
where
\begin{align}    
	\grad_{\V{r}_{i}}U_{\ell,\text{SOE}}^{\V{k}} &= - \frac{\pi q_{i}}{L_x L_y} \left[ \sum_{1 \leq j < i} q_{j} \grad_{\V{r}_{i}} \varphi_{\text{SOE}}^{\V{k}}(\V{r}_{i}, \V{r}_{j}) +  \sum_{i < j \leq N} q_{j} \grad_{\V{r}_{i}} \varphi_{\text{SOE}}^{\V{k}}(\V{r}_{j}, \V{r}_{i}) \right]\;,\\
	\grad_{\V{r}_{i}} U_{\ell,\text{SOE}}^{\V{0}} &= - \frac{2 \pi q_{i}}{L_x L_y} \left[ \sum_{1 \leq j < i} q_{j} \grad_{\V{r}_{i}} \varphi^{\bm{0}}_{\text{SOE}}(\bm{r}_{i}, \bm{r}_{j}) + \sum_{i < j \leq N} q_{j} \grad_{\V{r}_{i}} \varphi^{\bm{0}}_{\text{SOE}}(\bm{r}_{j}, \bm{r}_{i})\right]\;.
\end{align}
Using the approximation Eqs.~\eqref{eq::SOEphi},~\eqref{eq::dz_plus} and~\eqref{eq::dz_minus}, one can write the derivative in periodic directions as
\begin{equation}
	\begin{split}
		\partial_{\bm{\rho}_{i}} \varphi_{\text{SOE}}^{\V{k}}(\V{r}_{i}, \V{r}_{j}) & = \frac{\m{i} \V{k} e^{ \m{i} \V{k} \cdot \V{\rho}_{ij}}}{k} \left[\xi^{+}_M(k, z_{ij})+\xi^{-}_M(k, z_{ij})\right]\\
		&=\frac{2\alpha e^{-k^2/(4\alpha^2)}}{\sqrt{\pi}k} \m{i} \bm{k}e^{\m{i} \V{k} \cdot \V{\rho}_{ij}} \sum_{\ell = 1}^{M}  \frac{w_l}{\alpha^2 s_l^2 - k^2}\left( 2 \alpha s_l e^{-k z_{ij}} - 2 k e^{-\alpha s_l z_{ij}}\right),
	\end{split}
\end{equation}
and in~$z$ direction as
\begin{equation}\label{eq:z-der}
	\begin{split}
		\partial_{z_{i}} \varphi_{\text{SOE}}^{\V{k}}(\V{r}_{i}, \V{r}_{j}) & = \frac{e^{\m{i} \V{k} \cdot \V{\rho}_{ij}}}{k} \left[\partial_{z_{i}} \xi^{+}_M(k, z_{ij}) + \partial_{z_{i}} \xi^{-}_M(k, z_{ij})\right]\\
		&=\frac{2 \alpha e^{-k^2/(4\alpha^2)}}{\sqrt{\pi}}e^{ \m{i} \V{k} \cdot \V{\rho}_{ij}} \sum_{\ell = 1}^{M}  \frac{w_l}{\alpha^2 s_l^2 - k^2}\left( - 2 \alpha s_l e^{-k z_{ij}} + 2 \alpha s_l e^{- \alpha s_l z_{ij}}\right)\;.
	\end{split}
\end{equation}
The partial derivatives of zero-frequency mode with respect to the periodic directions are zero, and the SOE approximation of its $z$-derivative is given by
\begin{equation}\label{eq:dzphi_0}
	\begin{split}
		\partial_{z_{i}} \varphi^{\bm{0}}_{\text{SOE}}(\bm{r}_{i},\bm{r}_{j}) 
		& = \sum_{l=1}^{M} \frac{w_l}{\sqrt{\pi}} \partial_{z_{i}} \left[\frac{2z_{ij}}{s_l}+\left(\frac{1}{\alpha} - \frac{2z_{ij}}{s_l}\right)e^{-\alpha s_l z_{ij}}\right] \\
		& = \sum_{l=1}^{M} \frac{w_l}{\sqrt{\pi}} \left[ \frac{2}{s_l} - \left( s_l + \frac{2}{s_l} - 2 \alpha z_{ij} \right) e^{-\alpha s_l z_{ij}} \right]\;.
	\end{split}
\end{equation}

It is important to note that the computation of Fourier space forces using Eq.~\eqref{eq::Fi} follows a common recursive procedure with energy, since it has the same structure as given in Eq.~\eqref{eq::33}, and the overall cost for evaluating force on all~$N$ particles for each~$k$ point also amounts to~$O(N)$, and the resulting SOEwald2D method is summarized in Algorithm~\ref{alg:SOEwald2D}.

Moreover, Lemma~\ref{lem::forceerr} establishes the overall error on forces $\bm{F}_{i}$, and the proof follows an almost similar approach to what was done for the energy. %The proof is postponed to~\ref{app::forceerror}

\begin{lem}\label{lem::forceerr}
	The total error of force by the SOEwald2D is given by
	\begin{equation}
		\mathscr{E}_{\bm{F}_{i}} := \mathscr{E}_{\bm{F}_{\emph{s}}^{i}} + \mathscr{E}_{\bm{F}_{\emph{l}}^{i}} + \sum_{\bm{k}\neq \bm{0}} \mathscr{E}_{\bm{F}_{\emph{l}}^i, \emph{SOE}}^{\bm{k}} + \mathscr{E}_{\bm{F}_{\emph{l}}^{i},\emph{SOE}}^{\bm{0}}
	\end{equation}
	where the first two terms are the truncation error and provided in Proposition~\ref{prop::2.12}. The remainder terms 
	\begin{equation}
		\mathscr{E}_{\bm{F}_{\emph{l}}^i,\emph{SOE}}^{\bm{k}} := \bm{F}_{\emph{l}}^{\bm{k}, i} - \bm{F}_{\emph{l},\emph{SOE}}^{\bm{k}, i}, \quad\emph{and} \quad \mathscr{E}_{\bm{F}_{\emph{l}}^{i}, \emph{SOE}}^{\bm{0}} := \bm{F}_{\emph{l}}^{\bm{0}, i}-\bm{F}_{\emph{l}, \emph{SOE}}^{\bm{0}, i}
	\end{equation}
	are the error due to the SOE approximation as Eqs.~\eqref{eq::Fi}-\eqref{eq:z-der}. Given SOE parameters $w_l$ and $s_l$ along with the ideal-gas assumption, one has the following estimate:
	\begin{equation}
		\sum_{\bm{k}\neq\bm{0}} \mathscr{E}_{\bm{F}_{\emph{l}}^i, \emph{SOE}}^{ \bm{k}}\leq \sqrt{2}\lambda_D^2\alpha^2q_{i}^2\varepsilon,\quad\text{and}\quad \mathscr{E}_{\bm{F}_{\emph{l}}^{i},\emph{SOE}}^{\bm{0}}\leq \frac{4\sqrt{\pi}\lambda_D^2 (1+2\alpha)L_z}{L_xL_y}q_{i}^2\varepsilon.
	\end{equation}
\end{lem}



%the electrostatic potential satisfies the Poisson's equation
%\begin{equation}
%-\Delta \phi(\bm{r})=4\pi \rho_{q}(\bm{r}),\quad\rho_{q}(\bm{{r}})=q\rho_+g_{++}(\bm{r})-q\rho_{-}g_{+-}(\bm{r}),
%\end{equation}
%where the charge density are expressed in terms of the charge-charge correlation functions $g_{++}(\bm{r})=g_{--}(\bm{r})(\bm{r})$ and $g_{+-}(\bm{r})=g_{-+}(\bm{r})$, and $\rho_+=\rho_-=\rho/2$ are average densities of positive and negative ions. In the work of Debye and H$\ddot{\text{u}}$ckel, the correlation functions are approximated by $g_{ij}(\bm{r})=e^{-\beta q_j\phi_i(\bm{r})}$, 

%\section{The SOE approximation error of force}\label{app::forceerror}