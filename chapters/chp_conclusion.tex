\section{Conclusion}

Quasi-2D systems are ubiquitous in nature and play a crucial role in various fields, including materials science, biophysics, and electrochemistry. 
However, the simulation of quasi-2D systems is challenging due to the strong Coulomb interactions and the complex boundary conditions.
The challenges can be summarized as follows:
\begin{itemize}
    \item[1.] The reduced symmetry of the system leads to the breakdown of the traditional Ewald summation, especially for the systems with large aspect ratios.
    \item[2.] The dielectric confinement introduces additional complexity to the system, which requires careful treatment of the polarization effect, especially for the systems with strong polarizable interfaces.
    % \item[3.] The large system size and the need for high accuracy pose significant challenges for the computational resources.
\end{itemize}
Due to these challenges, existing methods are largely restricted to quasi-2D systems with moderate aspect ratios and weakly polarizable interfaces, limiting their broader applicability.

In this thesis, we have developed a class of novel fast algorithms for simulating quasi-2D Coulomb systems, making significant methodological advances in computational physics. 
The main contributions can be summarized in three key aspects:
\begin{itemize}
    \item[1.] Our method are mesh free and can reach the optimal $\mathcal{O}(N)$ complexity in both CPU and memory consumptions without relying on either FFT or FMM.
    \item[2.] Our method are able to handle the anisotropy of the system. Both analysis and numerical results validate that our method is not affected by the aspect ratio of the system.
    \item[3.] Our method are able to handle the dielectric confinement effect very efficiently, even for the case where the interfaces are stronly polarizable.
\end{itemize}
These advantages make our method a powerful tool for simulating quasi-2D Coulomb systems, and it is expected to have broad applications in various fields.


\section{Future Directions}

In the future, we are interested in extending our method to other interaction kernels, such as dipolar crystals, Stokesian fluids and Yukawa potentials~\cite{Messina2017PRL,Hou2009PRL}.
Furthermore, we aim to apply the RBE2D method to other complex systems, such as membrane-protein systems and battery-electrolyte systems, where confinement effects can significantly influence their properties.