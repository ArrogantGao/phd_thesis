\section{Background}

Confined quasi two dimensional (quasi-2D) Coulomb systems~\cite{mazars2011long}, which are macroscopic in two dimensions but with atomic-size thickness in the other, have caught much attention in studies of magnetic and liquid crystal films, super-capacitors, crystal phase transitions, dusty plasmas, ion channels, superconductive materials and quantum devices~\cite{kawamoto2002history, hille2001ionic, teng2003microscopic, Messina2017PRL, mazars2011long, saito2016highly, liu20192d}. 
Typically, such systems possess a nano-sized longitudinal thickness in the $z$ direction, achieved through confinement, bulk-like and modeled as periodic in the transverse $xy$ directions, hence endowed with an inherent multi-scale nature.
Due to the confinement effect, such systems can exhibit various interesting behaviors for future nanotechnologies; prototype examples include graphene~\cite{novoselov2004electric}, metal dichalcogenide monolayers~\cite{kumar2012tunable}, and colloidal monolayers~\cite{mangold2003phase}.

Among quasi-2D systems, charged particles confined by materials with different dielectric constants are of particular interest. 
The so-called \emph{dielectric confinement effect} can further lead to inhomogeneous screening, as compared to bulk systems, or even broken symmetries near interfaces~\cite{C0NR00698J,wang2016inhomogeneous,gao2024broken}.
These effects are crucial in a wide range of applications, including the behavior of water, electrolytes, and ionic liquids confined within thin films~\cite{raviv2001fluidity}, ion transport in nanochannels~\cite{zhu2019ion}, and the self-assembly of colloidal and polymer monolayers~\cite{kim2017imaging}.
In these systems, substrate materials used for nanoscale confinement can range from dielectric to metallic, and nowadays, electromagnetic metamaterials, which have been developed with permittivities that can take negative values~\cite{veselago1967electrodynamics, smith2004metamaterials, cheng2017tunable, xie2022recent, xu2020polyaniline} when excited by electromagnetic waves of specific frequencies.  
Noteworthy is that materials with \emph{negative static permittivity}  has drawn a considerable attention, though rarely seen, its existence has been predicted in materials such as metals and non-ideal plasma~\cite{Dolgov1918admissible,homes2001optical}. More recently, it has been experimentally achieved in a wide range of materials such as VO$_2$ films~\cite{kana2016thermally}, graphene~\cite{nazarov2015negative}, nanocolloids~\cite{shulman2007plasmalike}, and polymeric systems~\cite{yan2013negative}.
Interestingly, even for water, the perpendicular component of its tensorial dielectric function has been observed to be negative within sub-Angstrom distances from the surface by nano confinement~\cite{Kornyshev1996Static, Schlaich2016Water, Kornyshev2021Nonlocal}.


% To study the behavior of quasi-2D Coulomb systems, molecular dynamics (MD) simulations is one of the most powerful tools.
% By tracking the positions and velocities of all particles, MD can provide a detailed description of the system's static and dynamic properties.
% In MD simulations, the system is modeled as a collection of particles, which interact with each other through potential functions and the time evolution of the system is determined by the Newton's laws of motion.
% Thus, time cost of MD simulations is dominated by the calculation of the derivatives of the potential energy with respect to the particle positions, i.e. the force between particles.
% For the short-range potential such as the Lennard-Jones potential, the computational cost scales as $O(N)$ with the number of particles $N$ based on the neighbor list algorithm and the real space truncation.
% For long-range interactions such as the Coulomb interaction, the potential energy can be generally expressed as the lattice kernel summations in the form of

For numerical simulation of quasi-2D Coulomb systems, it is of great importance to evaluate lattice kernel summations in the form of
\begin{equation}\label{eq::phi12}
	\phi(\V x) = \sum_{\V{m}} \sum_{j = 1}^{N} \rho_j K(\V{x} - \V{y}_j + \V{m} \circ \V{L})\;,
\end{equation}
where~$\V{x}, \V{y}_j \in \mathbb{R}^3$ are $3$-dimensional vectors in a rectangular box $\Omega$ with $\V{L}$ the vector of its edge lengths, $\rho_j$ refers to the density or weight, $\V{m} \in \mathbb{Z}^2 \otimes \{0\}$ exerts periodicity in the first $2$ directions, ``$\circ$'' represents the Hadamard product, and~$K(\V{x})$ is the kernel function whose form depends on the interested physical problem. 
For homogeneous systems without dielectric confinements, the kernel function is given by $K(\V{x}) = \abs{\V{x}}^{-1}$.
For such long-range interactions, the computational cost scales as $O(N^2)$ in the naive pair-wise implementation, which greatly limits the size of the system that can be studied.


For fully-periodic or free-space Coulomb systems, various fast algorithms have been developed to reduce the computational cost, such as the fast multipole methods (FMM)~\cite{greengard1987fast,cheng1999fast,ying2004kernel}, fast Fourier transform (FFT) based Ewald-splitting methods~\cite{hockney2021computer,darden1993particle,essmann1995smooth} and the recently proposed random batch Ewald (RBE) method~\cite{jin2021random, liang2022superscalability,liang2024JCP}.
These fast algorithms have been successfully applied to large-scale simulations of Coulomb systems under various ensembles and reaching complexity of $O(N\log N)$ or even $O(N)$.
 
For quasi-2D systems,  numerous methods have also been developed over the past decades, mostly rely on modifications to the FMM and FFT-based Ewald methods to achieve $O(N\log N)$ or even $O(N)$ complexity.
One important work is the Ewald2D method~\cite{parry1975electrostatic}, which directly applies the Ewald-splitting technique for quasi-2D systems, it scales as $O(N^2)$ and is found to be slowly convergent due to the oscillatory nature of the Green's function for quasi-2D geometry.
To further accelerate the convergence of the quasi-2D lattice summation, more methods have been proposed, including the MMM2D method~\cite{arnold2002mmm2d}, the periodic FMM method~\cite{yan2018flexibly}, and the spectral Ewald method~\cite{lindbo2012fast}.
Instead of exactly solving the quasi-2D problem, alternative approaches are to approximate the quasi-2D summations by 3D periodic summations with correction terms, such as the Yeh--Berkowitz (YB) correction~\cite{yeh1999ewald} and the  {electric layer correction (ELC)~\cite{arnold2002electrostatics}.}
These correction methods effectively reduce the computational cost and are simple to implement, but lack control of accuracy.

The methods mentioned above encounter challenges when addressing sharp dielectric interfaces. In recent years, various strategies have been developed to effectively simulate quasi-2D systems under dielectric confinements.
A common approach is by combining the image charge method (ICM)~\cite{jackson1999classical,frenkel2023understanding}, where the polarization effects induced by dielectric planar interfaces are modeled through an infinite series of reflected image charges in homogeneous space.
Specifically, this approach truncates the infinitely reflected image charge series along the $z$-direction to a finite number of reflection layers. 
This reduction transforms the original dielectrically inhomogeneous system into a homogeneous one, albeit with an expanded dimension in $z$.
The remaining computational tasks can be effectively handled using the Ewald2D summation method~\cite{parry1975electrostatic,zhonghanhu2014JCTC}. While conceptually straightforward, this ICM-Ewald2D approach is computationally inefficient due to its $O(N^2)$ complexity.
Several alternative approaches have been developed over recent years, including the fast Fourier-Chebyshev spectral method~\cite{maxian2021fast,gao2024fast}, the harmonic surface mapping method~\cite{liang2020harmonic,liang2022hsma}, and boundary element methods~\cite{barros2014efficient,nguyen2019incorporating}.
When combined with the fast multipole method (FMM)~\cite{greengard1987fast,kohnke2020gpu}, or the fast Fourier-Chebyshev transform~\cite{trefethen2000spectral}, these methods can achieve computational complexities of $ O(N\log N)$ or even $ O(N)$.

Among existing approaches, a widely adopted technique involves reformulating the exact Ewald2D summation in terms of 3D Ewald summation, effectively approximating the quasi-2D problem with a fully periodic one.
Then, to eliminate unwanted contributions from periodic replicas along the $z$-axis, correction terms such as the Yeh-Berkowitz (YB) correction~\cite{yeh1999ewald,dos2015electrolytes} and the electric layer correction (ELC)~\cite{arnold2002electrostatics,tyagi2008electrostatic} have been introduced.
Its advantages include: (a) the use of 3D fast Fourier transform (FFT) to achieve $ O(N\log N)$ computational complexity~\cite{yuan2021particle,huang2024pmc}; 
(b) the straightforward incorporation of polarization contributions via image charges, as demonstrated in methods such as ICM-ELC~\cite{tyagi2008electrostatic}, ICM-Ewald3D~\cite{dos2015electrolytes}, and ICM-PPPM~\cite{yuan2021particle}; 
and (c) the minimal modifications needed to integrate these quasi-2D electrostatic solvers into mainstream software~\cite{ABRAHAM201519,thompson2021lammps}, where 3D periodic solvers have been extensively optimized.

% These approaches address the polarization effect either by introducing image charges for slab interfaces, effectively transforming the system into a homogeneous one (albeit with a considerable increase in thickness along the $z$-direction)~\cite{tyagi2007icmmm2d,tyagi2008electrostatic,dos2015electrolytes,yuan2021particle,liang2020harmonic}, or by numerically solving the Poisson equation with interface conditions~\cite{maxian2021fast, nguyen2019incorporating, ma2021modified}. 
% They all rely on either FMM or FFT to achieve $O(N)$ or $O(N\log N)$ complexity, respectively;

% We will not delve into a comprehensive review and comparison of these methods; instead, we will highlight two crucial aspects: 1) they all rely on either FMM or FFT to achieve   {$O(N)$ or $O(N\log N)$ complexity, respectively;} and 2) the computational cost significantly increases compared to the homogeneous case, especially for \emph{strongly confined} systems (i.e., with a large aspect ratio of simulation box). 
% In this scenario, FFT-based methods require a significant increase in   {resolution} to accommodate for the extended system thickness (i.e. zero-padding), while FMM-based methods require incorporating more near-field contributions and solving a possibly ill-conditioned linear system~\cite{pei2023fast}.

% However, different from the fully-periodic or free-space systems, quasi-2D systems possess a reduced symmetry, which gives rise to new phenomena, but also brings formidable challenges in both theory and computation.
% The first challenge comes from the involved \emph{long-range} interaction kernels, including but not limited to Coulomb and dipolar kernels in electrostatics, Oseen and Rotne-Prager-Yamakawa kernels in hydrodynamics and the static exchange-correlation kernels in density functional theory calculations. 
% The anisotropy of such systems poses extra challenges for simulations:
% (1) the periodic and non-periodic directions need to be handled separately due to their different boundary conditions and length scales;
% (2) the convergence properties of the lattice kernel summation requires careful consideration, which largely depend on the well-poseness of the underlying PDEs. 
% Another challenge comes from practical applications.
% To accurately determine the phase diagram of a many-body system may require thousands of simulation runs under different conditions~\cite{levin2002electrostatic}, each with billions of time steps to sample ensemble averages.
% Moreover, to eliminate the finite size effect, millions of free particles need to be simulated. Such large-scale simulations are especially required for quasi-2D systems, so as to accommodate its strong anisotropy, and resolving possible boundary layers forming near the confinement surfaces~\cite{mazars2011long}. 
% The cumulative impact of these considerations poses significant challenges for numerical simulations for quasi-2D systems. 

% To address these issues associated with the particle-based simulation of quasi-2D systems, a variety of numerical methods have been developed.
% Most of them also fall into two categories: 
% (1) Fourier spectral methods~\cite{lindbo2012fast,nestler2015fast,doi:10.1021/acs.jctc.3c01124, maxian2021fast}, where particles are first smeared onto grids, and subsequently the underlying PDE is solved in Fourier domain where fast Fourier transform (FFT) can be used for acceleration; 
% (2) adaptive tree-based methods, where fast multipole method (FMM)~\cite{greengard1987fast} or tree code~\cite{Barnes1986Nature} orginally proposed for free-space systems can be extended to quasi-2D systems by careful extension to match the partially-periodic boundary conditions~\cite{yan2018flexibly,liang2020harmonic}. 
% Alternative methods have also been proposed, such as the Lekner summation-based MMM2D method~\cite{arnold2002novel}, multilevel summation methods~\cite{doi:10.1021/ct5009075,greengard2023dual}, and correction-based approaches such as Ewald3DC~\cite{yeh1999ewald} and EwaldELC~\cite{arnold2002electrostatics}, which first solve a fully-periodic system and then add the partially-periodic correction terms. 
% By combining with either FFT or FMM, these methods achieve $O(N\log N)$ or even $O(N)$ complexity. 

However, the issue of large-scale simulation of quasi-2D systems is still far from settled.
A few challenges remains. First, FFT-based methods need extra techniques to properly handle the non-periodic direction, such as truncation~\cite{parry1975electrostatic}, regularization~\cite{nestler2015fast}, or periodic extension~\cite{lindbo2012fast}, which may lead to algebraic convergence or require extra zero-padding to guarantee accuracy. 
Recent advancements by Shamshirgar \emph{et al.}~\cite{shamshirgar2021fast}, combining spectral solvers with kernel truncation methods (TKM)~\cite{vico2016fast}, have reduced the zero-padding factor from $6$ to $2$~\cite{lindbo2012fast}, which still requires doubling the number of grids with zero-padding. 
% A similar reduction is also reported in the work of Maxian \emph{et al. }\cite{maxian2021fast}. 
Second, the periodization of FMM needs to encompass more near-field contributions from surrounding cells~\cite{yan2018flexibly,barnett2018unified}. The recently proposed 2D-periodic FMM~\cite{pei2023fast} may offer a promising avenue; however, it has not yet been extended to partially-periodic problems.
Finally, it is worth noting that most of the aforementioned issues will become more serious when the aspect ratio of the simulation box is large, in which case the Ewald series summation will converge much slower~\cite{arnold2002electrostatics}, and the zero-padding issue of FFT-based methods also becomes worse~\cite{maxian2021fast}.
Thus, efficient and accurate simulation of quasi-2D systems is still an open problem and a great challenge, especially for systems with large aspect ratios and strongly polarizable interfaces.


In this thesis, we focus on the study of confined quasi-2D Coulomb systems, including theoretical analysis, algorithm development and applications.
We provide comprehensive theoretical analysis including rigorous error estimates and complexity bounds, which are thoroughly validated through extensive numerical experiments.
We introduce a novel class of fast summation methods for quasi-2D Coulomb systems that synergistically combines the Ewald splitting technique with random batch sampling. 
Our proposed methods achieve optimal $O(N)$ computational complexity while maintaining high efficiency for strongly confined quasi-2D systems, even in challenging scenarios involving large aspect ratios and sharp dielectric interfaces. 
We also show a few applications of our methods to the simulation of quasi-2D Coulomb systems, including the study of broken symmetry in quasi-2D Coulomb systems and the simulation of electrolytes confined by dielectric interfaces.

\section{Thesis outlines}

The content of this thesis can be roughly divided into three parts, which are summarized as follows.

The first part focuses on the theoretical analysis about long-range interactions in quasi-2D Coulomb systems.

In Chapter~\ref{chp_preliminaries}, we introduce the physical model and mathematical notations for dielectric confined quasi-2D Coulomb systems.
Then we revisit the Ewald summation methods for both triply-periodic and doubly-periodic Coulomb systems.
We also provide a brief introduction to the random batch sampling method for fully-periodic Coulomb systems.

In Chapter~\ref{chp_icmewald2d}, we first introduce the Ewald summation method for dielectrically confined quasi-2D Coulomb systems, focusing on the ICM-Ewald2D method and its reformulation.
Then we provide detailed error analysis and guidelines for optimal parameter selection in these methods, along with extensive numerical validation of their performance for dielectrically confined quasi-2D systems.

In the second part, we introduce a novel class of fast summation methods for quasi-2D Coulomb systems.

In Chapter~\ref{chp_soewald2d}, we first introduce the kernel approximation technique called the sum-of-exponential (SOE) approximation, which approximates an arbitrary kernel function by a linear combination of exponential functions.
Then, based on the SOE approximation, we propose an algorithm named the sum-of-exponential Ewald2D (SOEwald2D) method, which utilizes the SOE approximation and the random batch sampling method to accelerate the Ewald2D summation for quasi-2D Coulomb systems without dielectric confinements.
Numerical results are provided to validate the accuracy and efficiency of the proposed method.

In Chapter~\ref{chp_rbe2d}, we focus on the quasi-2D Coulomb systems with dielectric confinements.
We first provide a rigorous error analysis for the combination of the image charge method and the Ewald summation formula.
Then, we introduce a novel algorithm called the random batch Ewald2D (RBE2D) method, which efficiently handles quasi-2D Coulomb systems with dielectric confinements while maintaining excellent scalability for large-scale simulations on modern supercomputing architectures.
Detailed numerical results are provided to validate the accuracy and efficiency of the proposed method.

In Chapter~\ref{chp_quasiewald}, we investigate quasi-2D systems with negative dielectric confinements, where the polarizable materials possess negative dielectric constants, which results in failure of the ICM-based methods.
% The previous methods based on the image charge method break down in such systems due to the unusual polarization effects.
To overcome the numerical challenges, we develop the quasi-Ewald method (QEM), which introduces a novel quasi-Ewald splitting technique.
We show that with properly singluarity subtraction, the QEM can be extended to handle the negative dielectric confinements.
Numerical results are also provided for validation.
% Using this method, we explore the behavior of negatively confined quasi-2D charged systems and discover remarkable spontaneous symmetry breaking (SSB) phenomena emerging in symmetrically charged binary particle systems.

In the third part (Chapter~\ref{chp_applications}), we demonstrate practical applications of our methods through two key studies of quasi-2D Coulomb systems. 
First, we conduct all-atom molecular dynamics simulations of the SPC/E water model under both standard and dielectrically confined conditions. 
These simulations validate the efficiency, accuracy, and scalability of our methods for large-scale real-world applications. 
Second, we investigate negatively confined electrolytes, revealing a fascinating spontaneous symmetry breaking (SSB) phenomenon that emerges in symmetrically charged binary particle systems purely through confinement effects.

Finally, Chapter~\ref{chp_conclusion} concludes this thesis with a comprehensive summary of our methodological contributions. 
We critically evaluate the strengths and limitations of our proposed algorithms, and outline promising future research directions for both theoretical developments and practical applications in computational physics and materials science.