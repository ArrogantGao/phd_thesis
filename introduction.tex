\section{Background}

% Molecular dynamics (MD) simulation is one of the most powerful tools for studying the behavior of solids and fluids in a rigorous and quantitative manner and has been widely used in many areas of physics, chemistry, biology, materials science, and related disciplines.
% By tracking the positions and velocities of all particles, MD can provide a detailed description of the system's static and dynamic properties.
% In MD simulations, the system is modeled as a collection of particles, which interact with each other through potential functions and the time evolution of the system is determined by the Newton's laws of motion.
% Thus, time cost of MD simulations is dominated by the calculation of the derivatives of the potential energy with respect to the particle positions, i.e. the force between particles.

% The potential can be generally classified into two categories: short-range and long-range.
% For the short-range potential such as the Lennard-Jones potential, the computational cost scales as $\mathcal O(N)$ with the number of particles $N$ based on the neighbor list algorithm and the real space truncation.
% For the long-range potential such as the Coulomb potential, the computational cost scales as $\mathcal O(N^2)$ in the naive implementation, which is prohibitive for large-scale systems.
% Fast algorithms are thus highly desirable for simulating the Coulomb systems.
% For isotropic Coulomb systems, deterministic algorithms with complexity of $\mathcal O(N)$ or $\mathcal O(N\log N)$ have been developed, which usually fall into one of the two categories: fast multipole methods (FMM)~\cite{greengard1987fast,cheng1999fast,ying2004kernel} and fast Fourier transform (FFT) based Ewald-splitting methods~\cite{hockney2021computer,darden1993particle,essmann1995smooth}.

Quasi two dimensional (quasi-2D) Coulomb systems~\cite{mazars2011long}, which are macroscopic in two dimensions but with atomic-size thickness in the other, have caught much attention in many areas of science and engineering. 
Typically, such systems possess a nano-sized longitudinal thickness in the $z$ direction, achieved through confinement, bulk-like and modeled as periodic in the transverse $xy$ directions, hence endowed with an  inherent multi-scale nature.
Due to the confinement effect, such systems can exhibit various interesting behaviors for future nanotechnologies;
prototype examples include graphene~\cite{novoselov2004electric}, metal dichalcogenide monolayers~\cite{kumar2012tunable}, and colloidal monolayers~\cite{mangold2003phase}.


To study the behavior of quasi-2D Coulomb systems, molecular dynamics (MD) simulations are one of the most powerful tools.
By tracking the positions and velocities of all particles, MD can provide a detailed description of the system's static and dynamic properties.
In MD simulations, the system is modeled as a collection of particles, which interact with each other through potential functions and the time evolution of the system is determined by the Newton's laws of motion.
Thus, time cost of MD simulations is dominated by the calculation of the derivatives of the potential energy with respect to the particle positions, i.e. the force between particles.
Unlike the short-range interaction such as the Lennard-Jones potential, the Coulomb interaction is long-range and scales as $\mathcal O(N^2)$ in the naive pair-wise implementation.


For fully-periodic or free-space Coulomb systems, various algorithms have been developed, such as the fast multipole methods (FMM)~\cite{greengard1987fast,cheng1999fast,ying2004kernel}, fast Fourier transform (FFT) based Ewald-splitting methods~\cite{hockney2021computer,darden1993particle,essmann1995smooth} and the recently proposed random batch Ewald (RBE) method~\cite{jin2021random, liang2022superscalability,liang2024JCP}.
These fast algorithms have been successfully applied to large-scale simulations of Coulomb systems under various ensembles and reaching complexity of $\mathcal O(N\log N)$ or even $\mathcal O(N)$.


However, for quasi-2D systems, the reduced symmetry gives rise to new phenomena, but also brings formidable challenges in both theory and computation.
The first challenge comes from the involved \emph{long-range} interaction kernels, including but not limited to Coulomb and dipolar kernels in electrostatics, Oseen and Rotne-Prager-Yamakawa kernels in hydrodynamics and the static exchange-correlation kernels in density functional theory calculations. 
% For fully-periodic or free-space systems, $\mathcal O(N)$ fast algorithms have been developed; but the field is still under developing for partially-periodic systems.
The anisotropy of such systems poses extra challenges for simulations:
(1) the periodic and non-periodic directions need to be handled separately due to their different boundary conditions and length scales;
(2) the convergence properties of the lattice kernel summation requires careful consideration, which largely depend on the well-poseness of the underlying PDEs. 
Another challenge comes from practical applications.
To accurately determine the phase diagram of a many-body system may require thousands of simulation runs under different conditions~\cite{levin2002electrostatic}, each with billions of time steps to sample ensemble averages.
Moreover, to eliminate the finite size effect, millions of free particles need to be simulated. Such large-scale simulations are especially required for quasi-2D systems, so as to accommodate its strong anisotropy, and resolving possible boundary layers forming near the confinement surfaces~\cite{mazars2011long}. 
The cumulative impact of these considerations poses significant challenges for numerical simulations for quasi-2D systems. 

To address these issues associated with the particle-based simulation of quasi-2D systems, a variety of numerical methods have been developed.
Most of them fall into two categories: 
(1) Fourier spectral methods~\cite{lindbo2012fast,nestler2015fast,doi:10.1021/acs.jctc.3c01124, maxian2021fast}, where particles are first smeared onto grids, and subsequently the underlying PDE is solved in Fourier domain where fast Fourier transform (FFT) can be used for acceleration; 
(2) adaptive tree-based methods, where fast multipole method (FMM)~\cite{greengard1987fast} or tree code~\cite{Barnes1986Nature} orginally proposed for free-space systems can be extended to quasi-2D systems by careful extension to match the partially-periodic boundary conditions~\cite{yan2018flexibly,liang2020harmonic}. 
Alternative methods have also been proposed, such as the Lekner summation-based MMM2D method~\cite{arnold2002novel}, multilevel summation methods~\cite{doi:10.1021/ct5009075,greengard2023dual}, and correction-based approaches such as Ewald3DC~\cite{yeh1999ewald} and EwaldELC~\cite{arnold2002electrostatics}, which first solve a fully-periodic system and then add the partially-periodic correction terms. 
By combining with either FFT or FMM, these methods achieve $\mathcal{O}(N\log N)$ or even $\mathcal{O}(N)$ complexity. 

However, the issue of large-scale simulation of quasi-2D systems is still far from settled.
A few challenges remains. First, FFT-based methods need extra techniques to properly handle the non-periodic direction, such as truncation~\cite{parry1975electrostatic}, regularization~\cite{nestler2015fast}, or periodic extension~\cite{lindbo2012fast}, which may lead to algebraic convergence or require extra zero-padding to guarantee accuracy. 
Recent advancements by Shamshirgar \emph{et al.}~\cite{shamshirgar2021fast}, combining spectral solvers with kernel truncation methods (TKM)~\cite{vico2016fast}, have reduced the zero-padding factor from $6$ to $2$~\cite{lindbo2012fast}, which still requires doubling the number of grids with zero-padding. 
% A similar reduction is also reported in the work of Maxian \emph{et al. }\cite{maxian2021fast}. 
Second, the periodization of FMM needs to encompass more near-field contributions from surrounding cells~\cite{yan2018flexibly,barnett2018unified}. The recently proposed 2D-periodic FMM~\cite{PEI2023111792} may offer a promising avenue; however, it has not yet been extended to partially-periodic problems.
Finally, it is worth noting that most of the aforementioned issues will become more serious when $L_z\ll \min\{L_x, L_y\}$, in which case the Ewald series summation will converge much slower~\cite{arnold2002electrostatics}, and the zero-padding issue of FFT-based methods also becomes worse~\cite{maxian2021fast}.
Thus, efficient and accurate simulation of quasi-2D systems is still an open problem and a great challenge.

\section{Thesis Outlines}

