% TeX'ing this file requires that you have AMS-LaTeX 2.0 installed
% as well as the rest of the prerequisites for REVTeX 4.0
%
% See the REVTeX 4 README file
% It also requires running BibTeX. The commands are as follows:
%
%  1)  latex apssamp.tex
%  2)  bibtex apssamp
%  3)  latex apssamp.tex
%  4)  latex apssamp.tex
%
%\documentclass[prb,showkeys,preprintnumbers,amsmath,amssymb, 11pt]{revtex4}
%\documentclass[preprint,showpacs,showkeys,preprintnumbers,amsmath,amssymb]{revtex4}

% Some other (several out of many) possibilities
%\documentclass[preprint,aps]{revtex4}
%\documentclass[aps, two column, amsmath,amssymb,floatfix]{revtex4}
%\documentclass[showkeys,showpacs,amsmath,amssymb,onecolumn,superscriptaddress,prl]{revtex4-1}% Physical Review B  
\documentclass[aps,prl,reprint,showpacs,floatfix,superscriptaddress, onecolumn, 12pt]{revtex4-2}

\usepackage{amsmath,amsthm,amssymb}
\usepackage{graphicx}% Include figure files
\usepackage{dcolumn}% Align table columns on decimal point
\usepackage{bm}% bold math
\usepackage{color}
\usepackage{epsfig}
\usepackage{multirow}
\usepackage{mathrsfs}
\usepackage{hyperref}
\usepackage{cleveref}
\usepackage{epstopdf}
\usepackage{subfigure}
\usepackage{autobreak}

%Macros for mathematical notations

\newcommand{\V}[1]{\boldsymbol{#1}} %# vector
\newcommand{\M}[1]{\boldsymbol{#1}} %# matrix
\newcommand{\Set}[1]{\mathbb{#1}} %# set
\newcommand{\D}[1]{\Delta#1} %# \D{t} for time step size
\renewcommand{\d}[1]{\delta#1} %# \d{t} for small increment
\newcommand{\norm}[1]{\left\Vert #1\right\Vert } % norm
\newcommand{\abs}[1]{\left|#1\right|} %abs

\newcommand{\grad}{\M{\nabla}} %gradient
\newcommand{\av}[1]{\left\langle #1\right\rangle } %take average

\newcommand{\sM}[1]{\M{\mathcal{#1}}} %matrix in mathcal font
\newcommand{\dprime}{\prime\prime} % double prime
%\global\long\def\i{\iota}
%\renewcommand{\i}{\iota} %i for imaginary unit
%\renewcommand{\i}{\mathsf i} %i for imaginary unit
\newcommand{\follows}{\quad\Rightarrow\quad} %=>
\newcommand{\eqd}{\overset{d}{=}} %=^d
\newcommand{\spe}[1]{\mathscr{#1}}  %important quantities in mathscr font
\newcommand{\eps}{\epsilon}

\newcommand{\ar}[1]{{\color{blue}{Authors' response: #1}}} % for authors' response
\newcommand{\blue}[1]{{\color{blue}{#1}}} % for authors' Response
\newcommand{\red}[1]{{\color{red}{#1}}} % for authors' Revised
\newcommand{\xuanzhao}[1]{{\color{red}{XZ: #1}}} % for referee's comment
\usepackage{xcolor}
% Define deep green color using RGB
\definecolor{deepgreen}{RGB}{0, 100, 0}
\newcommand{\jiuyang}[1]{{\color{deepgreen}{JY: #1}}}

\begin{document}
\preprint{Preprint}

\title{Response to committee members' comments}
\author{}
% \date{}

\maketitle

\noindent Dear committee members,

Please find enclosed a revised version of my thesis. I would like to thank the committee members for their valuable comments and suggestions, which greatly help improve the quality of this thesis. I have carefully revised the thesis. A file identical to the revised version of the thesis, but with changes marked again in blue is attached. Below, I first summarize the main changes, followed by itemized responses and corrections to all the committee members' comments.

%titled ``Accurate Error Estimates and Optimal Parameter Selection in Ewald Summation for Dielectrically Confined Coulomb Systems''.
 %We believe these revisions have improved the clarity and quality of our manuscript.

\vspace{1em}

\textbf{Major changes}

\blue{
    \begin{enumerate}
        \item 
    \end{enumerate}
}

\vspace{1em}

\textbf{Response to the comments}

\begin{enumerate}

\item In the introduction, boundary element methods are suggested to include to provide a more comprehensive introcution to the topic.

\ar{
    Thank you for the suggestion, it is true that boundary element methods are a powerful tool to solve the Poisson equation with complex boundary conditions. A brief introduction to boundary element methods has been added to the introduction (page 3 and 4).
}

\item page 15: ``introduce it applications" should be ``introduce its applications"

\ar{
    The typo has been corrected.
}

\item Abstract should be revised to clarify not only algorithms but also some new physical insights obtained.

\ar{
    In the revised version, I have added a new paragraph to the abstract to clarify the new physical insights obtained.
}

\item Some figures in the thesis don't have units.

\item Proposition 2.2.7 is not well-written, the work ``well-defined" needs clarification.

\item The author developed several useful packages, which are good reference for junior students to get started. I hope the they can be briefly introduced in the thesis, maybe in the appendix.

\end{enumerate}



\end{document}